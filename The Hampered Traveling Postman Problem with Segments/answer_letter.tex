\documentclass{article}
\usepackage[most]{tcolorbox}
\usepackage[a4paper,top=1in, bottom=1.25in, left=1.25in, right=1.25in]{geometry}
\usepackage{amsmath}
\usepackage{amsthm}
\usepackage{capt-of}
\usepackage{graphicx}
\usepackage{caption,subcaption}
\usepackage{url}
\usepackage{multirow}
\usepackage{enumerate}
%\usepackage{tikz}
\usepackage{epstopdf}% To incorporate .eps illustrations using PDFLaTeX, etc.
%\usepackage{subfigure}% Support for small, `sub' figures and tables
\usepackage{nameref}
\usepackage{zref-xr,zref-user}
\zxrsetup{toltxlabel=true, tozreflabel=false}
%\zexternaldocument*[original:]{TSC}
\usepackage{xcite}
\usepackage{hyperref}
\usepackage{ulem}
%\externalcitedocument[org:]{TSC}

%\usepackage[table]{xcolor}
%\usepackage{color}
%\usepackage{colortbl}

\definecolor{Gray}{gray}{0.9}
\newcommand{\coldscr}{\cellcolor{Gray}}

\newcommand{\initresponses}{\newcounter{pointcounter}}

\newenvironment{reviewer}{\setcounter{pointcounter}{1}}{}

\newenvironment{mybiblio}{\small}{}


%\newcommand{\point}{{\textsl{\thepointcounter}. \stepcounter{pointcounter} #1}}

%\newcommand{\point}[1]{\medskip \noindent \text{{\selectfont \thepointcounter} \stepcounter{pointcounter} #1}}

\newcommand{\point}{\text{{\selectfont \thepointcounter} \stepcounter{pointcounter}}}


\newcommand{\mynum}[1]{^{(#1)}}
\newcommand{\myi}{\mynum{i}}
\newcommand{\mym}{\mynum{m}}
\newcommand{\mymi}{\mynum{m,i}}
\newcommand{\myMi}{\mynum{M,i}}
\newcommand{\myq}{\mynum{q,i}}
\newcommand{\myzeroi}{\mynum{0,i}}
\newcommand{\myduei}{^{(i)\;2}}
\newcommand{\JP}[1]{{\color{blue}#1}}
\newcommand{\LA}[1]{{\color{red}#1}}
\newcommand{\TP}[1]{{\color{orange}#1}}
\begin{comment}
	\usetikzlibrary{shapes.geometric,backgrounds,
		positioning-plus,node-families,calc}
	\tikzset{
		basic box/.style = {
			shape = rectangle,
			align = center,
			draw  = #1,
			fill  = #1!25,
			rounded corners},
		header node/.style = {
			Minimum Width = 0.4cm,
			font          = \strut\scriptsize\ttfamily,
			text depth    = +0pt,
			fill          = white,
			draw},
		header/.style = {%
			inner ysep = +1.5em,
			append after command = {
				\pgfextra{\let\TikZlastnode\tikzlastnode}
				node [header node] (header-\TikZlastnode) at (\TikZlastnode.north) {#1}
				node [span = (\TikZlastnode)(header-\TikZlastnode)]
				at (fit bounding box) (h-\TikZlastnode) {}
			}
		},
		hv/.style = {to path = {-|(\tikztotarget)\tikztonodes}},
		vh/.style = {to path = {|-(\tikztotarget)\tikztonodes}},
		fat blue line/.style = {ultra thick, blue}
	}
	
	\tikzstyle{dummy} = [rectangle, text width=0.1em, draw=white, white,
	minimum width=0.1em, minimum height=3em, opacity=0.0]
	
	\tikzstyle{mycircle} = [circle, draw=black, black, text width=1em, minimum height=1em]
	
	\tikzstyle{mydiamond} = [diamond, aspect=2, draw=gray, fill=gray!25, text width=6em, minimum height=1em]
	
	\tikzstyle{startend} =  [rectangle, font=\strut\scriptsize\ttfamily, text depth=+0pt, fill=white, draw=black]
\end{comment}

\hyphenation{dif-fe-rent}

\title{CAIE-D-23-00610
	\\
	"The Hampered Travelling Salesman Problem with Neighbourhoods"}
\author{Answer to Reviewers' Comments}
\begin{document}
	\maketitle
	%\begin{abstract}
	%\todo[inline,color=green!50]
	%{Abstract changed to adapt to format indicated in
		%guidelines to authors. Text has beeen changed to
		%reflect the update of Section 2 and Discussion.}
	%\lipsum[1]
	%\end{abstract}
	%\section{Introduction}
	%Really et al. (2010)
	%\todo[color=blue!40]{Added citation}
	%said some important suff.\lipsum[2]
	%\lipsum[3]
	
	We wish to thank the editors and reviewers for their valuable comments and advices which allowed us to further improve the quality and presentation of our article through this revision.
	
	We revised the manuscript taking into account all the suggestions of Reviewers 1 and 2. We highlighted in blue all the changes in the revised manuscript. In the following, we report our changes inside the coloured textboxes.
	%{\bf We outlined in bold each change made in this new version of the paper}.
	\initresponses
	
	\begin{reviewer}
		
		\begin{tcolorbox}[breakable,enhanced,coltitle=black,colback=red!75!black,colframe=red!75!black,borderline={1pt}{0pt}{black},boxrule=0pt]
			\textbf{Reviewer 1}
		\end{tcolorbox}
		
		\begin{itshape}
			This is an interesting and difficult problem. There are, however, many parts
			expressed in vague terms. Furthermore, the authors often rely on results from the
			literature. I think these results need to be stated and their interpretation has to be
			explicitly written in the studied framework , not by just citing references.
		\end{itshape}
		
		\begin{tcolorbox}[breakable,enhanced,coltitle=black,colback=red!5!white,colframe=red!75!black,title=\textbf{Answer R1.\point},borderline={1pt}{0pt}{black},boxrule=0pt]

		\end{tcolorbox}
		
		\begin{itshape}
			Assumption A3: Which kind of union is operated between two overlapping barriers. Is it a convex union? But in case of segments, the resulting barrier is not a segment!
		\end{itshape}
		
		\begin{tcolorbox}[breakable,enhanced,coltitle=black,colback=red!5!white,colframe=red!75!black,title=\textbf{Answer R1.\point},borderline={1pt}{0pt}{black},boxrule=0pt]

		\end{tcolorbox}
		
		\begin{itshape}
			Second Order Cone seems to be an important property but here a very few information is given about it. Please, try to give more details in Subsection 3.1 (definition, properties...)
		\end{itshape}
		
		\begin{tcolorbox}[breakable,enhanced,coltitle=black,colback=red!5!white,colframe=red!75!black,title=\textbf{Answer R1.\point},borderline={1pt}{0pt}{black},boxrule=0pt]

		\end{tcolorbox}
		
		\begin{itshape}
			Page 14: The first sentence is about computing the dominating sets, the construction algorithm must be specified.
		\end{itshape}
		
		\begin{tcolorbox}[breakable,enhanced,coltitle=black,colback=red!5!white,colframe=red!75!black,title=\textbf{Answer R1.\point},borderline={1pt}{0pt}{black},boxrule=0pt]


		\end{tcolorbox}
		
		\begin{itshape}
			The single-commodity flow formulation is considered for the Steiner Travelling salesman
			problem for two reasons: the fictitious vertices associated to barriers and the shortest path
			problem formulation. This is an example where the explanation of this motivation is scattered
			throughout many paragraphs. Try to summerize and to shorten the text.
		\end{itshape}
		
		\begin{tcolorbox}[breakable,enhanced,coltitle=black,colback=red!5!white,colframe=red!75!black,title=\textbf{Answer R1.\point},borderline={1pt}{0pt}{black},boxrule=0pt]

		\end{tcolorbox}
		
		\begin{itshape}
			Page 20: The difference between H-TSPHN and H-TSPN is given but the authors have to describe the $\alpha$-constraints, specially the product of continuous variables obtained for this case.
		\end{itshape}
		
		\begin{tcolorbox}[breakable,enhanced,coltitle=black,colback=red!5!white,colframe=red!75!black,title=\textbf{Answer R1.\point},borderline={1pt}{0pt}{black},boxrule=0pt]

		\end{tcolorbox}
		
		
		\begin{itshape}
			\underline{Minor corrections:}
			\begin{itemize}
				\item \textbf{Page 3:} 
				\begin{itemize}
					\item 2nd paragraph, last line : ``obStacles''.
					\item last paragraph: 6 sections.
				\end{itemize}
				\item \textbf{Page 5:} Use ``endnodes'' of an edge instead of vertices.
				\item \textbf{Page 11:} 
				\begin{itemize} 
					\item In the definition of $E_X$ replace $\wedge$ by a comma.
					\item Figures 1 and 2: add a), b), c) for each case.
				\end{itemize}
				\item \textbf{Page 12:} In the mathematical formulation: the last of constraints is ($N$-C) or ($U$-C)? Make it clear.
				\item \textbf{Page 23}: Figure 7: Keep the same notation as in the algorithm 1: barrier $B_1$ and corresponding points denoted $P^1_{B_1}$ instead of $P^1_1$ for example. 
				\item \textbf{Page 27}: 
				\begin{itemize}
					\item Proof of Proposition 6: Define exactly the visibility graph, its vertices and edges.
					\item Algorithm 3: Is step 2 defined by Algorithm 4? same thing for step 3 and Algorithm 5? Explain the connections between the different algorithms.
				\end{itemize}
				\item \textbf{Page 29}:
				\begin{itemize}
					\item In the second experiment, the number of instances is ten, isn’t it?
					\item Indicate that the unit time is the second.
					\item Tables 2 and 3, column $|\mathcal B|$: are the values average?
				\end{itemize}
			\end{itemize}
			
		\end{itshape}

				
	\end{reviewer}
	
	\begin{reviewer}
		
		\begin{tcolorbox}[breakable,enhanced,coltitle=black,colback=green!75!black,colframe=green!75!black,borderline={1pt}{0pt}{black},boxrule=0pt]
			\textbf{Reviewer 2}
		\end{tcolorbox}
		
		\begin{itshape}
			This article deals with the study of variants of the TSP: the TSP with consideration of neighbourhoods and barriers. According to the authors, this is the first time that this variant of the problem has been studied. This variant of the TSP is far removed from the original problem. It is sufficient to pass close to a city to visit it (in its neighbourhood) and there are obstacles or barriers (in the form of impassable segments) that require detours to get from one point to another. This study is particularly motivated by surveillance or distribution applications using drones.
			
			The paper is structured in 6 parts and studies three variants: the H-SPPN (Hampered Shortest Path Problem with Neighbourhoods), the H-TSPHN (Hampered Traveling Salesman Problem with Hidden Neighbourhoods) and the H-TSPN (Hampered Traveling Salesman Problem with Neighbourhoods), which is a relaxed form of the previous one.
			
			Overall, this article is of an excellent scientific standard. It presents a mathematical model of the three variants, a proof of their complexity (the first variant is polynomial, the second and therefore the third are NP-complete), a strengthening of the mathematical formulations and a comprehensive experimental study that demonstrates the effectiveness of the approach adopted. I recommend that the paper be accepted, considering the minor remarks made in the more detailed comments that follow.
		\end{itshape}
		
		\begin{tcolorbox}[breakable,enhanced,coltitle=black,colback=green!5!white,colframe=green!75!black,title=\textbf{Answer R2.\point},borderline={1pt}{0pt}{black},boxrule=0pt]!
			
		\end{tcolorbox}
		
		\begin{itshape}
			Section 1 is a long introduction in which the variants considered are presented and positioned in relation to the literature. The state of the art seems complete and is well structured. The authors make an effort to illustrate their problem with concrete potential applications, which is much appreciated. The presentation of the plan mistakenly mentions that the article is structured in 8 sections instead of 6.
		\end{itshape}
		
		\begin{tcolorbox}[breakable,enhanced,coltitle=black,colback=green!5!white,colframe=green!75!black,title=\textbf{Answer R2.\point},borderline={1pt}{0pt}{black},boxrule=0pt]

		\end{tcolorbox}
		
		\begin{itshape}
			Section 2 introduces important preliminary notions and notations that will be used in the remainder of the paper. The three problems are represented in the form of a graph whose vertices are points chosen from the neighbourhoods or the endpoints of the barriers, and the edges connect two vertices provided that they do not intersect any obstacles. The graphs for the three problems are presented separately, which makes them easier and clearer to read. The beginning of section 2 (presentation of hypotheses and notations) can be numbered within a sub-section 2.1.
		\end{itshape}
		
		\begin{tcolorbox}[breakable,enhanced,coltitle=black,colback=green!5!white,colframe=green!75!black,title=\textbf{Answer R2.\point},borderline={1pt}{0pt}{black},boxrule=0pt]

		\end{tcolorbox}
		
		\begin{itshape}
			Section 3 is a very long section (pages 7 to 20) devoted to modelling problems in the form of a mixed nonlinear programme (MINLP). This is the heart of the article. This part is very well written, and the reader is led progressively through the modelling of the various problems and the proof of their complexity. Well-chosen figures illustrate the subject and make it easier to understand.
		\end{itshape}
		
		\begin{tcolorbox}[breakable,enhanced,coltitle=black,colback=green!5!white,colframe=green!75!black,title=\textbf{Answer R2.\point},borderline={1pt}{0pt}{black},boxrule=0pt]

		\end{tcolorbox}
		
		\begin{itshape}
			Section 3 is a very long section (pages 7 to 20) devoted to modelling problems in the form of a mixed nonlinear programme (MINLP). This is the heart of the article. This part is very well written, and the reader is led progressively through the modelling of the various problems and the proof of their complexity. Well-chosen figures illustrate the subject and make it easier to understand.
		\end{itshape}
				
		\begin{tcolorbox}[breakable,enhanced,coltitle=black,colback=green!5!white,colframe=green!75!black,title=\textbf{Answer R2.\point},borderline={1pt}{0pt}{black},boxrule=0pt]
			
		\end{tcolorbox}
		
		\begin{itshape}
			Section 4 strengthens the formulation presented in the previous section by focusing on three proposals: removing unnecessary edges, adjusting the value of big-M, and fixing some variables. The results presented in the next section will show the interest of these strengthening. It might be interesting to know the marginal contribution of each of these proposals to improving the solver's behaviour.
		\end{itshape}
		
		\begin{tcolorbox}[breakable,enhanced,coltitle=black,colback=green!5!white,colframe=green!75!black,title=\textbf{Answer R2.\point},borderline={1pt}{0pt}{black},boxrule=0pt]
			
		\end{tcolorbox}
		
		\begin{itshape}
			The results are presented in section 5. The datasets are presented first, with two classes of instances. The first is completely generated and contains instances of 20 to 100 neighbourhoods to visit. The second is generated using instances from the literature for the "close-enough TSP", and adding obstacles. The results show the effectiveness of the solver (and also its limitations)
			\begin{itemize}
			\item I invite the authors to make the instances they have generated available to the community.
			\item It might be interesting to build an (or several) instance(s) from a real city map.
			\item It might be interesting for the reader to visualise an optimal solution
			\end{itemize}
		\end{itshape}
		
		\begin{tcolorbox}[breakable,enhanced,coltitle=black,colback=green!5!white,colframe=green!75!black,title=\textbf{Answer R2.\point},borderline={1pt}{0pt}{black},boxrule=0pt]
			
		\end{tcolorbox}
		
		\begin{itshape}
			The conclusion gives both theoretical and practical perspectives to this work.
		\end{itshape}
		
		\begin{tcolorbox}[breakable,enhanced,coltitle=black,colback=green!5!white,colframe=green!75!black,title=\textbf{Answer R2.\point},borderline={1pt}{0pt}{black},boxrule=0pt]
			
		\end{tcolorbox}
		
		\begin{itshape}
			A general comment: The H-SPPN is a polynomial variant that is not considered at all in the experimental part. At the end of the reading, I did not understand the interest of this variant and its contribution to the solution of the two other variants. I recommend that the authors improve or better explain how knowing how to solve the H-SPPN can help to solve the two other problems.
		\end{itshape}
		
		\begin{tcolorbox}[breakable,enhanced,coltitle=black,colback=green!5!white,colframe=green!75!black,title=\textbf{Answer R2.\point},borderline={1pt}{0pt}{black},boxrule=0pt]
			
		\end{tcolorbox}
		
	\end{reviewer}
	
	
\end{document}
