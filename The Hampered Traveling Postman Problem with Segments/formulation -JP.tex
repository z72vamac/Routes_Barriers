\documentclass[a4paper]{elsarticle}
\usepackage[utf8]{inputenc}
\usepackage[T1]{fontenc}
%\usepackage[spanish]{babel}
\usepackage{amsmath}
\usepackage{amsfonts}
\usepackage{amssymb}
\usepackage{graphicx}
\usepackage{mathtools,amssymb}
\usepackage{subfigure}
\usepackage{optidef}
\usepackage{xcolor}
\usepackage{amsthm}
\usepackage{comment}
\usepackage{tikz}
\usepackage{pgfplots}
\usepackage{mathrsfs}
\usepackage{float}
\usepackage[linesnumbered,ruled,vlined]{algorithm2e}
%\usepackage[noend]{algpseudocode}
%\usepackage{ulem}
\usepackage[margin=1in]{geometry}
\usepackage{enumitem}
\usepackage{xspace}
\usepackage{array}
\setlength{\extrarowheight}{.5ex}


\usepackage{threeparttable}

\usepackage[utf8]{inputenc}

\title{The Traveling Postman Problem with Linear Barriers}
\author{carlosvalverdemartin }
\date{February 2021}

\DeclareMathOperator*{\argmax}{arg\,max}
\DeclareMathOperator*{\argmin}{arg\,min}


\newcommand{\SPPN}{{\sf{H-SPPN}\xspace }}
\newcommand{\TSPN}{{\sf{H-TSPN}\xspace }}
\newcommand{\TSPVN}{{\sf{H-TSPVN}\xspace }}
\newcommand{\B}{{\mathcal B}}
\newcommand{\VB}{{V^{}_{\mathcal B}}}
\newcommand{\EB}{{E^{}_{\mathcal B}}}
\newcommand{\VS}{{V^{}_{S}}}
\newcommand{\ES}{{E^{}_{S}}}
\newcommand{\VT}{{V^{}_{T}}}
\newcommand{\ET}{{E^{}_{T}}}
\newcommand{\VN}{{V^{}_{\mathcal N}}}
\newcommand{\EN}{{E^{}_{\mathcal N}}}

\newtheorem{remark}{Remark}
\newtheorem{notation}{Notation}

\newtheorem{prop}{Proposition}

\definecolor{armygreen}{rgb}{0.19, 0.53, 0.43}
\definecolor{atomictangerine}{rgb}{1.0, 0.6, 0.4}
\newcommand{\JP}[1]{{\color{armygreen}#1}}
\newcommand{\CV}[1]{{\color{atomictangerine}#1}}


\begin{document}
\section{Description of the Problem}\label{section:description}
RThis section describes the two problems that are considered in this paper: the Hampered Shortest Path Problem with Neighborhoods \SPPN \ and the Hampered Traveling Salesman Problem with Neighborhoods \TSPN. 

%In \SPPN, we have a source neighborhood $N_S\subset\mathbb R^2$ and a target neighborhood $N_T\subset\mathbb R^2$, that we assume to be convex sets and a set $\mathcal B$ of opened line segments located in general position that plays the role of barriers that the drone cannot cross, i.e., the drone can visit the endpoints of these barriers but it cannot be located in the interior points of the barriers. Note that, it is rational to assume that if there are two barriers that have a portion of them in common, it is only considered the smallest line segment that contains both barriers. The aim of the \SPPN is to find the best pair of points $(P_{S}, P_{T})\in N_S\times N_T$ in the source and target neighborhoods that minimize the length of the path that joins both points without crossing any barrier of $\mathcal B$. In this problem, it is implicitly assumed that there is not a rectilinear path to go from $N_S$ to $N_T$, otherwise, the problem becomes straightforward and the solution is the minimum distance between both neighborhoods. 
In \SPPN, we have a source neighborhood $N_S\subset\mathbb R^2$ and a target neighborhood $N_T\subset\mathbb R^2$, that we assume to be second order cone representable sets and a set $\mathcal B$ of line segments that play the role of barriers that the drone cannot cross. In this model, we state the following assumptions:

\begin{enumerate}[label=\textbf{A\arabic*},ref=\textbf{A\arabic*}]
\item \label{A1}The line segments of $\mathcal B$ are located in general position, i.e., the endpoints of these segments are not aligned. Although it is possible to model the most general case, \JP{one can always to slightly modify one of the endpoints so that the segments are in general position.}
\item The line segments of $\mathcal B$ are opened, that is, it is possible that the drone visits the endpoints of the segments, but entering  in the interior points of them is not allowed. \JP{Observe that without loss of genrality, we can always slightly enlarge these segments to make them opened.}
\item If there are two barriers that have a common portion of them, it is only considered the smallest line segment that contains both barriers.
\item \label{A4}There is no a rectilinear path to go from $N_S$ to $N_T$. Otherwise, the problem becomes straightforward and the solution is the minimum distance between both neighborhoods.
\end{enumerate}

The aim of the \SPPN is to find the best pair of points $(P_{S}, P_{T})\in N_S\times N_T$ in the source and target neighborhoods that minimize the length of the path that joins both points without crossing any barrier of $\mathcal B$ and assuming \ref{A1}-\ref{A4}.



The \TSPN \ is an extension of the \SPPN \ where the neighborhood set $\mathcal N$ is considered to play the role of source and target  in the \SPPN \ and moreover, a set of given targets must be visited. The aim of the \TSPN \  is to seek the shortest tour that visits each neighborhood $N\in\mathcal N$ exactly once without crossing any barrier $B\in\mathcal B$ and assuming again \ref{A1}-\ref{A4}. \CV{Note that, in this case, it may be interesting to consider this problem without taking into account the assumption \ref{A4}. This more general version will be called the Hampered Traveling Salesman Problem with Visible Neighborhoods \TSPVN.}

\JP{Figure \ref{fig:initialdata} shows an example of the problem \SPPN \ that is considered. In the left picture, the blue neighborhood represents the source, the green one the target and the red line segments show the barriers that the drone cannot cross. In the right picture, an instance of the \TSPN \  is shown, where the neighborhood are balls and the barriers are, again, the red line segments.}

\pgfplotsset{compat=1.15}
\usetikzlibrary{arrows}
\definecolor{qqffqq}{rgb}{0,1,0}
\definecolor{qqqqff}{rgb}{0,0,1}
\definecolor{ffqqqq}{rgb}{1,0,0}
\definecolor{bfffqq}{rgb}{0.7490196078431373,1,0}
\definecolor{ffxfqq}{rgb}{1,0.4980392156862745,0}
\definecolor{qqffqq}{rgb}{0,1,0}
\definecolor{qqqqff}{rgb}{0,0,1}
\definecolor{ffqqqq}{rgb}{1,0,0}


\begin{figure}
	\centering
%	\caption{Problem data of the \SPPN, \TSPHN \ and \TSPN}
	\begin{minipage}{.3\textwidth}
		\caption{\SPPN \ instance}\label{fig:spp}
		\centering
		\begin{tikzpicture}[line cap=round,line join=round,>=triangle 45,x=1cm,y=1cm,scale=0.45]
		\begin{axis}[
		x=0.1cm,y=0.1cm,
		axis lines=middle,
		%ymajorgrids=true,
		%xmajorgrids=true,
		xmin=-5,
		xmax=105,
		ymin=-5,
		ymax=105,
		xtick={-30,-20,...,150},
		ytick={-20,-10,...,110},]
		\clip(-30.53772939574263,-29.679586779798747) rectangle (159.53639158056973,116.68410916363342);
		\draw [line width=1pt,color=ffqqqq] (20,80)-- (40,30);
		\draw [line width=1pt,color=ffqqqq] (70,95)-- (40,70);
		\draw [line width=1pt,color=ffqqqq] (95,60)-- (60,70);
		\draw [line width=1pt,color=ffqqqq] (60,50)-- (90,10);
		\draw [line width=1pt,color=ffqqqq] (10,70)-- (20,50);
		\draw [rotate around={0:(20,10)},line width=1pt,color=qqqqff,fill=qqqqff,fill opacity=0.25] (20,10) ellipse (1cm and 1cm);
		\draw [rotate around={0:(90,90)},line width=1pt,color=qqffqq,fill=qqffqq,fill opacity=0.25] (90,90) ellipse (0.5cm and 0.5cm);
		\begin{scriptsize}
		\draw [color=ffqqqq] (20,80) circle (2.5pt);
		\draw [color=ffqqqq] (40,30) circle (2.5pt);
		\draw [color=ffqqqq] (70,95) circle (2.5pt);
		\draw [color=ffqqqq] (40,70) circle (2.5pt);
		\draw [color=ffqqqq] (95,60) circle (2.5pt);
		\draw [color=ffqqqq] (60,70) circle (2.5pt);
		\draw [color=ffqqqq] (60,50) circle (2.5pt);
		\draw [color=ffqqqq] (90,10) circle (2.5pt);
		\draw [color=ffqqqq] (10,70) circle (2.5pt);
		\draw [color=ffqqqq] (20,50) circle (2.5pt);
		\end{scriptsize}
		\end{axis}
		\end{tikzpicture}
	\end{minipage}
	\hspace{0.2 in}
	\begin{minipage}{.3\textwidth}
		\caption{\TSPHN \ instance}\label{fig:tsph}
		\centering
		\begin{tikzpicture}[line cap=round,line join=round,>=triangle 45,x=1cm,y=1cm,scale=0.45]
		\begin{axis}[
		x=0.1cm,y=0.1cm,
		axis lines=middle,
		%ymajorgrids=true,
		%xmajorgrids=true,
		xmin=-5,
		xmax=105,
		ymin=-5,
		ymax=105,
		xtick={-30,-20,...,150},
		ytick={-20,-10,...,110},]
		\clip(-30.53772939574263,-29.679586779798747) rectangle (159.53639158056973,116.68410916363342);
		\draw [line width=1pt,color=ffqqqq] (20,80)-- (40,30);
		\draw [line width=1pt,color=ffqqqq] (70,95)-- (40,70);
		\draw [line width=1pt,color=ffqqqq] (95,60)-- (60,70);
		\draw [line width=1pt,color=ffqqqq] (60,50)-- (90,10);
		\draw [line width=1pt,color=ffqqqq] (10,70)-- (20,50);
		\draw [rotate around={0:(20,10)},line width=1pt,color=qqqqff,fill=qqqqff,fill opacity=0.25] (20,10) ellipse (1cm and 1cm);
		\draw [rotate around={0:(90,90)},line width=1pt,color=qqffqq,fill=qqffqq,fill opacity=0.25] (90,90) ellipse (0.5cm and 0.5cm);
		\draw [rotate around={0:(35,85)},line width=1pt,color=ffxfqq,fill=ffxfqq,fill opacity=0.25] (35,85) ellipse (0.9cm and 0.9cm);
		\draw [rotate around={0:(85,40)},line width=1pt,color=bfffqq,fill=bfffqq,fill opacity=0.25] (85,40) ellipse (1.1cm and 1.1cm);
		\begin{scriptsize}
		\draw [color=ffqqqq] (20,80) circle (2.5pt);
		\draw [color=ffqqqq] (40,30) circle (2.5pt);
		\draw [color=ffqqqq] (70,95) circle (2.5pt);
		\draw [color=ffqqqq] (40,70) circle (2.5pt);
		\draw [color=ffqqqq] (95,60) circle (2.5pt);
		\draw [color=ffqqqq] (60,70) circle (2.5pt);
		\draw [color=ffqqqq] (60,50) circle (2.5pt);
		\draw [color=ffqqqq] (90,10) circle (2.5pt);
		\draw [color=ffqqqq] (10,70) circle (2.5pt);
		\draw [color=ffqqqq] (20,50) circle (2.5pt);
		\end{scriptsize}
		\end{axis}
		\end{tikzpicture}

	\end{minipage}
	\hspace{0.2 in}
	\begin{minipage}{.3\textwidth}
		\centering
		\caption{\TSPN \ instance}\label{fig:tsp}
		\begin{tikzpicture}[line cap=round,line join=round,>=triangle 45,x=1cm,y=1cm,scale=0.45]
			\begin{axis}[
				x=0.1cm,y=0.1cm,
				axis lines=middle,
				%ymajorgrids=true,
				%xmajorgrids=true,
				xmin=-5,
				xmax=105,
				ymin=-5,
				ymax=105,
				xtick={-30,-20,...,150},
				ytick={-20,-10,...,110},]
				\clip(-30.53772939574263,-29.679586779798747) rectangle (159.53639158056973,116.68410916363342);
				\draw [line width=1pt,color=ffqqqq] (70,95)-- (40,70);
				\draw [line width=1pt,color=ffqqqq] (95,60)-- (60,70);
				\draw [line width=1pt,color=ffqqqq] (60,50)-- (90,10);
				\draw [line width=1pt,color=ffqqqq] (10,70)-- (20,50);
				\draw [rotate around={0:(20,10)},line width=1pt,color=qqqqff,fill=qqqqff,fill opacity=0.25] (20,10) ellipse (1cm and 1cm);
				\draw [rotate around={0:(90,90)},line width=1pt,color=qqffqq,fill=qqffqq,fill opacity=0.25] (90,90) ellipse (0.5cm and 0.5cm);
				\draw [rotate around={0:(35,85)},line width=1pt,color=ffxfqq,fill=ffxfqq,fill opacity=0.25] (35,85) ellipse (0.9cm and 0.9cm);
				\draw [rotate around={0:(85,40)},line width=1pt,color=bfffqq,fill=bfffqq,fill opacity=0.25] (85,40) ellipse (1.1cm and 1.1cm);
				\begin{scriptsize}
					\draw [color=ffqqqq] (70,95) circle (2.5pt);
					\draw [color=ffqqqq] (40,70) circle (2.5pt);
					\draw [color=ffqqqq] (95,60) circle (2.5pt);
					\draw [color=ffqqqq] (60,70) circle (2.5pt);
					\draw [color=ffqqqq] (60,50) circle (2.5pt);
					\draw [color=ffqqqq] (90,10) circle (2.5pt);
					\draw [color=ffqqqq] (10,70) circle (2.5pt);
					\draw [color=ffqqqq] (20,50) circle (2.5pt);
				\end{scriptsize}
			\end{axis}
		\end{tikzpicture}
	\end{minipage}
	\label{fig:initialdata}
\end{figure}

\section{MINLP Formulations}\label{section:formulations}

This section  introduces a Mixed Integer Non-Linear Programming formulation for the problems described in Section \ref{section:description}. First of all, we set the constraints that check whether two segments intersect. Then, we present the conic programming representation of the neighborhoods and the distance. Finally, the formulation for the \SPPN \ is described and the model \TSPN \ is explained as an extension of the previous problem.

\CV{

First of all, we introduce the decision variables that represent the problem. These variables are summarized in Table \ref{table:variables}.

\begin{table}[h!]
\centering
\caption{Summary of decision variables used in the mathematical programming model}
\label{table:variables}
\begin{tabular}{|cl|l}
\cline{1-2}
\multicolumn{2}{|l|}{\textbf{Binary Decision Variables}} &  \\ \cline{1-2}
\multicolumn{1}{|l|}{\textbf{Name}} & \textbf{Description} &  \\ \cline{1-2}
\multicolumn{1}{|c|}{$\alpha(P|\overline{QQ'})$} & \begin{tabular}[c]{@{}l@{}}1, if the determinant $\det(P|\overline{QQ'})$ is positive,\\ 0, otherwise.\end{tabular} &  \\ \cline{1-2}
\multicolumn{1}{|c|}{$\beta(\overline{PP'}|\overline{QQ'})$} & \begin{tabular}[c]{@{}l@{}}1, if the determinants $\det(P|\overline{QQ'})$ and $\det(P'|\overline{QQ'})$ have the same sign,\\ 0, otherwise.\end{tabular} &  \\ \cline{1-2}
\multicolumn{1}{|c|}{$\gamma(\overline{PP'}|\overline{QQ'})$} & \begin{tabular}[c]{@{}l@{}}1, if  the determinants $\det(P|\overline{QQ'})$ and $\det(P'|\overline{QQ'})$ are both positive,\\ 0, otherwise.\end{tabular} &  \\ \cline{1-2}
\multicolumn{1}{|c|}{$\delta(\overline{PP'}|\overline{QQ'})$} & \begin{tabular}[c]{@{}l@{}}1, if the line segments $\overline{PP'}$ and $\overline{QQ'}$ intersect,\\ 0, otherwise.\end{tabular} &  \\ \cline{1-2}
\multicolumn{1}{|c|}{$\epsilon(\overline{PP'})$} & \begin{tabular}[c]{@{}l@{}}1, if the line segment $\overline{PP'}$ does not cross any barrier,\\ 0, otherwise.\end{tabular} &  \\ \cline{1-2}
\multicolumn{1}{|c|}{$y(PQ)$} & \begin{tabular}[c]{@{}l@{}}1, if the edge $(P, Q)$ is selected in the solution of the model,\\ 0, otherwise.\end{tabular} &  \\ \cline{1-2}
\multicolumn{2}{|l|}{\textbf{Continuous Decision Variables}} & \multicolumn{1}{c}{\textbf{}} \\ \cline{1-2}
\multicolumn{1}{|l|}{\textbf{Name}} & \textbf{Description} &  \\ \cline{1-2}
\multicolumn{1}{|c|}{$P_N$} & Coordinates representing the point selected in the neighborhood $N$. &  \\ \cline{1-2}
\multicolumn{1}{|c|}{$d(PQ)$} & Euclidean distance between the points $P$ and $Q$. &  \\ \cline{1-2}
\multicolumn{1}{|c|}{$g(PQ)$} & Amount of commodity passing through the edge $(P, Q)$. &  \\ \cline{1-2}
\end{tabular}
\end{table}
}

\subsection{Checking whether two segments intersect}
Let $\overline{PP'}$ and $\overline{QQ'}$ be two line segments and the goal is to check whether they intersect. The following well-known computational geometry result can be used to analyze their relative position:

\newcommand{\segment}[2]{\overline{#1#2}}
\newcommand{\determinant}[3]{\det({#1|\overline{#2#3}})}


\begin{remark}\label{rem:determinants}
Let $\overline{PP'}$ and $\overline{QQ'}$ be two different line segments. Let also denote 
$$
\determinant{P}{Q}{Q'}=\det\left(\begin{array}{c|c} \overrightarrow{PQ} & \overrightarrow{PQ'}\end{array}\right):=\det\left( \begin{array}{cc}  Q_x-P_x & Q'_x-P_x \\ Q_y-P_y & Q'_y-P_y \end{array}\right)$$ 
the determinant whose arguments are $P=(P_x,P_y)$, $Q=(Q_x,Q_y)$ and $Q'=(Q'_x,Q'_y)$. If
\begin{equation*}
\normalfont{\text{sign}}\left(\determinant{P}{Q}{Q'}\right) = \normalfont{\text{sign}}\left(\determinant{P'}{Q}{Q'}\right)
\quad
\text{or}
\quad
\normalfont{\text{sign}}\left(\determinant{Q}{P}{P'}\right) = \normalfont{\text{sign}}\left(\determinant{Q'}{P}{P'}\right)
,
\end{equation*}
then $\overline{PP'}$ and $\overline{QQ'}$ do not intersect.
\end{remark}

In the following, we are going to model this condition by introducing some binary variables that check the sign of the determinants, the equality of signs, and the disjunctive condition.

\newcommand{\LS}[3]{L(#1|\overline{#2#3})}
\newcommand{\US}[3]{U(#1|\overline{#2#3})}
\newcommand{\alphamas}[3]{\alpha(#1|\overline{#2#3})}
\newcommand{\alphamenos}[3]{\alpha^{-}(#1|#2#3)}
%\newcommand{\alphacero}[3]{\alpha^{0\,}(#1|#2#3)}
\newcommand{\alphapunto}[3]{\alpha^{\cdotp}(#1|#2#3)}

To model the sign of a determinant defined by the points $P$, $Q$ and $Q'$, we introduce the binary variable $\alphamas{P}{Q}{Q'}$, that assumes the value one if $\determinant{P}{Q}{Q'}$ is positive and zero, otherwise. Note that the case when the determinant is null does not need to be considered, because the segments are located in general position.

It is possible to represent the sign condition by including the following constraints:
\begin{equation}\tag{$\alpha$-C}\label{eq:alphaC}
\left[1-\alphamas{P}{Q}{Q'}\right]\LS{P}{Q}{Q'}\leq\determinant{P}{Q}{Q'}\leq \US{P}{Q}{Q'}\:\alphamas{P}{Q}{Q'},
\end{equation}

\noindent where $\LS{P}{Q}{Q'}$ and $\US{P}{Q}{Q'}$ are a lower and an upper bound for the value of the determinant, respectively. If the determinant is positive, then $\alphamas{P}{Q}{Q'}$ must be one to make the second inequality feasible. The analogous case happens if the determinant is not positive.

\newcommand{\betamas}[4]{\beta(\overline{#1#2}|\overline{#3#4})}
%\newcommand{\betamenos}[4]{\beta^{-}(#1#2|#3#4)}
%\newcommand{\betacero}[4]{\beta^{0\,}(#1#2|#3#4)}
%\newcommand{\betapunto}[4]{\beta^{\cdotp}(#1#2|#3#4)}

Now, to check if the two determinants $\determinant{P}{Q}{Q'}$ and $\determinant{P'}{Q}{Q'}$ have the same sign, we introduce the binary variable $\betamas{P}{P'}{Q}{Q'}$, that is one if $\determinant{P}{Q}{Q'}$ and $\determinant{P'}{Q}{Q'}$ have the same sign,  and zero otherwise.

\newcommand{\gammaprod}[4]{\gamma(\overline{#1#2}|\overline{#3#4})}

Hence, the $\beta$ variable can be represented by the equivalence constraint of the $\alpha$ variables
\begin{align*}%\tag{$\beta$-C}\label{eq:betaC}
\betamas{P}{P'}{Q}{Q'}&=\alphamas{P}{Q}{Q'}\alphamas{P'}{Q}{Q'} + \left[1-\alphamas{P}{Q}{Q'}\right]\left[1-\alphamas{P'}{Q}{Q'}\right].
\end{align*}
This condition can be equivalently written using the auxiliary binary variable $\gammaprod{P}{P'}{Q}{Q'}$ is  that models the product of the $\alpha$ variables:
\begin{align*}\tag{$\beta$-C}\label{eq:betaC}
\betamas{P}{P'}{Q}{Q'}&=2\gammaprod{P}{P'}{Q}{Q'} -\alphamas{P}{Q}{Q'}-\alphamas{P'}{Q}{Q'}+1,
\end{align*}
We observe that $\gammaprod{P}{P'}{Q}{Q'}$ can be linearized by using the following constraints:
\begin{align*}\tag{$\gamma$-C}\label{eq:gammaC}
\gammaprod{P}{P'}{Q}{Q'} & \leq \alphamas{P}{Q}{Q'},\\
\gammaprod{P}{P'}{Q}{Q'} & \leq \alphamas{P'}{Q}{Q'},\\
\gammaprod{P}{P'}{Q}{Q'} & \geq \alphamas{P}{Q}{Q'} + \alphamas{P'}{Q}{Q'} - 1.
\end{align*}
 

\newcommand{\deltacheck}[4]{\delta(\overline{#1#2}|\overline{#3#4})}

Finally, we need to check whether there exists any coincidence in the sign of the determinants, so we define the binary variable $\deltacheck{P}{P'}{Q}{Q'}$ that is one if $\segment{P}{P'}$ and $\segment{Q}{Q'}$ do not intersect and zero, otherwise. This condition can be modelled by using these disjunctive constraints:
\begin{equation*}\tag{$\delta$-C}\label{eq:deltaC}
\frac{1}{2}\left[\betamas{P}{P'}{Q}{Q'}+\betamas{Q}{Q'}{P}{P'}\right]\leq\deltacheck{P}{P'}{Q}{Q'}\leq 2\left[\betamas{P}{P'}{Q}{Q'}+\betamas{Q}{Q'}{P}{P'}\right].
\end{equation*}
Indeed, the above constraint states that if there exists a sign coincidence, then $\deltacheck{P}{P'}{Q}{Q'}$ is one to satisfy the first constraint, and the second one is always fullfilled. However, if the sign of the determinants is not the same, then the second constraint is active and $\deltacheck{P}{P'}{Q}{Q'}$ is null.

\subsection{Conic programming constraints in the models}
In both considered  problems, namely \SPPN \ and \TSPN, there exist two second-order cone constraints that model the distance between the pair of points $P$ and $Q$, as well as  the representation of the neighborhoods where the points can be selected.

\newcommand{\dvar}[2]{d(#1#2)}

For the former case, we introduce the nonnegative continuous variable $\dvar{PQ}$ that represents the distance between $P$ and $Q$:


\begin{equation*}\tag{d-C}\label{eq:dC}
\|P - Q\|\leq \dvar{P}{Q},\quad\forall (P,Q)\in E.
\end{equation*}

For the latter case, since we are assuming that the neighborhoods are second-order cone (SOC) representable, they can be expressed by means of the constraints:

\begin{equation*}\tag{N-C}\label{eq:nC}
 P^{}_N\in N \Longleftrightarrow
  \|A_N^i P_N^{} + b_N^i\| \leq (c_N^i)^T P_N^{} + d_N^i,\quad i=1,\ldots,nc_N, \\
\end{equation*}
%\begin{equation}\label{C-C}\tag{$\mathcal{C}$-C}
%    \|B_ix + b_i\|\leq c_i^Tx + d_i,\quad i=1,\ldots,N,
%\end{equation}
where $A_N^i, b_N^i, c_N^i$ and $d_N^i$ are parameters of the constraint $i$ and $nc_N$ denotes the number of constraints that appear in the block associated to the neighborhood $N$.

It is remarkable that these inequalities can model the special case of linear constraints (for $A_N^{i}, b_N^i\equiv 0$), ellipsoids and hyperbolic constraints (see \cite{Lobo1998} for more information).

\textcolor{red}{Puede ser muy interesante el caso 3D pero quizás en otro trabajo, no?}

\subsection{A formulation for the \SPPN}
The main goal of the \SPPN\  is to solve a shortest path problem in the undirected graph induced by the endpoints of the barriers and the neighborhoods. To state the model,  we define the following sets:
\begin{itemize}
\item $\VS=\{P_S\}$: set composed by the point selected in the source neighborhood $N_S$.
\item $\VB=\{P^1_B, P^2_B:B=\overline{P^1_B P^2_B}\in \mathcal B\}$: set of vertices that come from the endpoints of the barriers in the problem.
\item $\VT=\{P^{}_T\}$: set composed by the point selected in the target neighborhood $N_T$.
\item $\ES=\{(P_S, P^i_{B}):P^i_B\in V_\B\text{ and } \overline{P_SP^i_B}\cap B''=\emptyset,\forall B''\in\B,\:i=1,2\}$: set of edges formed by the line segments that join the point selected in the source neighborhood and every endpoint in the barriers that do not cross any barrier in $\B$.
\item $\EB=\{(P^{i}_B, P^{j}_{B'}):P^i_B, P^j_{B'}\in \VB \text{ and } \overline{P^i_B P^j_{B'}}\cap B''=\emptyset,\:\forall B''\in\mathcal B,\:i, j=1,2\}$: set of edges formed by the line segments that join two vertices of $V_{\mathcal B}$ and do not cross any barrier in $\B$.
\item $\ET=\{(P^{}_T, P^i_{B}):P^i_B\in V_\B\text{ and } \overline{P^{}_TP^i_B}\cap B''=\emptyset,\forall B''\in\B,\:i=1,2\}$: set of edges formed by the line segments that join the point selected in the target neighborhood and every endpoint in the barriers that do not cross any barrier in $\B$.
\textcolor{red}{Definimos tambien la arista que pueda unir a los dos neighborhoods o asumimos que no hay un camino que los una? Depende del caso que consideremos el problema es convexo o no.}
\end{itemize} 

At this point, we can define the graph $G= (V, E)$ induced by the barriers and the neighborhoods, where $V=\VS\cup \VB\cup\VT$ and $E=\ES\cup\EB \cup\ET$. It is interesting to note that this graph can be split into two parts: a fixed graph $G_\B=(\VB,\EB)$ whose edges can be computed by using the Remark \ref{rem:determinants} and the sets $\VS$, $\ES$, $\VT$ and $\ET$ that depend on where the points $P_S$ and $P^{}_T$ are located as shown in Figure \ref{fig:graph}.  The figures show how the graph $G$ is generated. The blue dashed line segments represent the edges of $\ES$, the green ones, the edges of $\ET$ and the red dashed lines, the edges of $\EB$. A special case that can be remarked occurs when the neighborhoods are points. In that case, the induced graph is completely fixed and it is only necessary to find which edges are included by keeping in mind that there can not have crossings. This idea is exploited in Subsection \ref{section:reformulation}.

%\textcolor{red}{Me estoy dando cuenta que nos pueden decir que para qué tenemos en cuenta todo el conjunto si realmente se pueden coger los puntos de la frontera mas cerca a cada uno de los puntos por donde puede salir de las barreras. Es cierto que solo ocurre con el SPP. En el caso del TSP esto no es cierto.}

\pgfplotsset{compat=1.15}
\usetikzlibrary{arrows}
\definecolor{qqffqq}{rgb}{0,1,0}
\definecolor{qqqqff}{rgb}{0,0,1}
\definecolor{ffqqqq}{rgb}{1,0,0}
\begin{figure}[h!]
\centering
\begin{minipage}{.4\linewidth}
	\centering
\caption{Generation of the visibility graph $\GSPP$. Case 1}
\begin{tikzpicture}[line cap=round,line join=round,>=triangle 45,x=1cm,y=1cm, scale=0.5]
\begin{axis}[
x=0.1cm,y=0.1cm,
axis lines=middle,
xmin=-5,
xmax=105,
ymin=-5,
ymax=105,
xtick={-30, -20,...,100},
ytick={-30, -20,...,100},]
\clip(-30.537729395742623,-36.63351803502969) rectangle (160.8609499148995,109.73017790840248);
\draw [line width=1pt,color=ffqqqq] (20,80)-- (40,30);
\draw [line width=1pt,color=ffqqqq] (70,95)-- (40,70);
\draw [line width=1pt,color=ffqqqq] (95,60)-- (60,70);
\draw [line width=1pt,color=ffqqqq] (60,50)-- (90,10);
\draw [line width=1pt,color=ffqqqq] (10,70)-- (20,50);
\draw [line width=1pt,dashed,color=ffqqqq] (10,70)-- (20,80);
\draw [line width=1pt,dashed,color=ffqqqq] (20,50)-- (40,30);
\draw [line width=1pt,dashed,color=ffqqqq] (90,10)-- (40,30);
\draw [line width=1pt,dashed,color=ffqqqq] (20,80)-- (40,70);
\draw [line width=1pt,dashed,color=ffqqqq] (20,80)-- (70,95);
\draw [line width=1pt,dashed,color=ffqqqq] (40,70)-- (40,30);
\draw [line width=1pt,dashed,color=ffqqqq] (40,70)-- (60,70);
\draw [line width=1pt,dashed,color=ffqqqq] (70,95)-- (95,60);
\draw [line width=1pt,dashed,color=ffqqqq] (95,60)-- (90,10);
\draw [line width=1pt,dashed,color=ffqqqq] (60,50)-- (40,30);
\draw [line width=1pt,dashed,color=ffqqqq] (40,30)-- (60,70);
\draw [line width=1pt,dashed,color=ffqqqq] (60,70)-- (70,95);
\draw [line width=1pt,dashed,color=ffqqqq] (20,50)-- (20,80);
\draw [rotate around={0:(20,10)},line width=1pt,color=qqqqff,fill=qqqqff,fill opacity=0.25] (20,10) ellipse (1cm and 1cm);
\draw [rotate around={0:(90,90)},line width=1pt,color=qqffqq,fill=qqffqq,fill opacity=0.25] (90,90) ellipse (0.5cm and 0.5cm);
\draw [line width=1pt,dashed,color=ffqqqq] (10,70)-- (40,30);
\draw [line width=1pt,dashed,color=ffqqqq] (60,50)-- (60,70);
\draw [line width=1pt,dashed,color=ffqqqq] (60,50)-- (95,60);
\draw [line width=1pt,dashed,color=ffqqqq] (90,10)-- (60,70);
\draw [line width=1pt,dashed,color=qqqqff] (12.786085173820345,16.92527493177169)-- (10,70);
\draw [line width=1pt,dashed,color=qqqqff] (12.786085173820345,16.92527493177169)-- (20,50);
\draw [line width=1pt,dashed,color=qqqqff] (12.786085173820345,16.92527493177169)-- (40,30);
\draw [line width=1pt,dashed,color=qqqqff] (12.786085173820345,16.92527493177169)-- (90,10);
\draw [line width=1pt,dashed,color=qqffqq] (89.83150923646912,88.53724455912725)-- (70,95);
\draw [line width=1pt,dashed,color=qqffqq] (89.83150923646912,88.53724455912725)-- (40,70);
\draw [line width=1pt,dashed,color=qqffqq] (89.83150923646912,88.53724455912725)-- (60,70);
\draw [line width=1pt,dashed,color=qqffqq] (89.83150923646912,88.53724455912725)-- (95,60);
\draw [color=qqqqff](3.9007872968296593,8.815389811658244) node[anchor=north west] {$\mathbf{N_S}$};
\draw [color=qqffqq](95.46088215737039,101.20333363115502) node[anchor=north west] {$\mathbf{N_T}$};
\draw (87.18239256780974,95.5) node[anchor=north west] {$\mathbf{P_T}$};
\draw (5.722055006533001,19.742996069878295) node[anchor=north west] {$\mathbf{P_S}$};
\draw [line width=1pt,dashed,color=ffqqqq] (40,70)-- (60,50);
\draw [line width=1pt,dashed,color=ffqqqq] (60,50)-- (20,80);
\draw [line width=1pt,dashed,color=ffqqqq] (20,80)-- (90,10);
\draw [color=ffqqqq](68.5,53.8) node[anchor=north west] {$\mathbf{G_{\mathcal B}=(V_{\mathcal B}, E_{\mathcal B})}$};
\draw [color=qqffqq](68,90.11015758114377) node[anchor=north west] {$\mathbf{E_T}$};
\draw [color=qqqqff](21.1200456431158,33.9819981639226) node[anchor=north west] {$\mathbf{E_S}$};
\draw [line width=1pt,dashed,color=ffqqqq] (40,70)-- (95,60);
\draw [line width=1pt,dashed,color=ffqqqq] (40,70)-- (90,10);
\draw [line width=1pt,dashed,color=ffqqqq] (40,30)-- (70,95);
\begin{scriptsize}
\draw [color=ffqqqq] (20,80) circle (2.5pt);
\draw [color=ffqqqq] (40,30) circle (2.5pt);
\draw [color=ffqqqq] (70,95) circle (2.5pt);
\draw [color=ffqqqq] (40,70) circle (2.5pt);
\draw [color=ffqqqq] (95,60) circle (2.5pt);
\draw [color=ffqqqq] (60,70) circle (2.5pt);
\draw [color=ffqqqq] (60,50) circle (2.5pt);
\draw [color=ffqqqq] (90,10) circle (2.5pt);
\draw [color=ffqqqq] (10,70) circle (2.5pt);
\draw [color=ffqqqq] (20,50) circle (2.5pt);
\draw [fill=qqqqff] (12.786085173820345,16.92527493177169) circle (2.5pt);
\draw [fill=qqffqq] (89.83150923646912,88.53724455912725) circle (2.5pt);
\end{scriptsize}
\end{axis}
\end{tikzpicture}
\label{fig:graph1}
\end{minipage}
\hspace{1 cm}
\begin{minipage}{.4\linewidth}
	\centering
\caption{Generation of the visibility graph $\GSPP$. Case 2}
\begin{tikzpicture}[line cap=round,line join=round,>=triangle 45,x=1cm,y=1cm, scale=0.5]
\begin{axis}[
	x=0.1cm,y=0.1cm,
	axis lines=middle,
	xmin=-5,
	xmax=105,
	ymin=-5,
	ymax=105,
	xtick={-30,-20,...,160},
	ytick={-30,-20,...,100},]
\clip(-30.537729395742623,-36.63351803502969) rectangle (160.8609499148995,109.73017790840248);
\draw [line width=1pt,color=ffqqqq] (20,80)-- (40,30);
\draw [line width=1pt,color=ffqqqq] (70,95)-- (40,70);
\draw [line width=1pt,color=ffqqqq] (95,60)-- (60,70);
\draw [line width=1pt,color=ffqqqq] (60,50)-- (90,10);
\draw [line width=1pt,color=ffqqqq] (10,70)-- (20,50);
\draw [line width=1pt,dashed,color=ffqqqq] (10,70)-- (20,80);
\draw [line width=1pt,dashed,color=ffqqqq] (20,50)-- (40,30);
\draw [line width=1pt,dashed,color=ffqqqq] (90,10)-- (40,30);
\draw [line width=1pt,dashed,color=ffqqqq] (20,80)-- (40,70);
\draw [line width=1pt,dashed,color=ffqqqq] (20,80)-- (70,95);
\draw [line width=1pt,dashed,color=ffqqqq] (40,70)-- (40,30);
\draw [line width=1pt,dashed,color=ffqqqq] (40,70)-- (60,70);
\draw [line width=1pt,dashed,color=ffqqqq] (70,95)-- (95,60);
\draw [line width=1pt,dashed,color=ffqqqq] (95,60)-- (90,10);
\draw [line width=1pt,dashed,color=ffqqqq] (60,50)-- (40,30);
\draw [line width=1pt,dashed,color=ffqqqq] (40,30)-- (60,70);
\draw [line width=1pt,dashed,color=ffqqqq] (60,70)-- (70,95);
\draw [line width=1pt,dashed,color=ffqqqq] (20,50)-- (20,80);
\draw [rotate around={0:(20,10)},line width=1pt,color=qqqqff,fill=qqqqff,fill opacity=0.25] (20,10) ellipse (1cm and 1cm);
\draw [rotate around={0:(90,90)},line width=1pt,color=qqffqq,fill=qqffqq,fill opacity=0.25] (90,90) ellipse (0.5cm and 0.5cm);
\draw [line width=1pt,dashed,color=ffqqqq] (10,70)-- (40,30);
\draw [line width=1pt,dashed,color=ffqqqq] (60,50)-- (60,70);
\draw [line width=1pt,dashed,color=ffqqqq] (60,50)-- (95,60);
\draw [line width=1pt,dashed,color=ffqqqq] (90,10)-- (60,70);
\draw [line width=1pt,dashed,color=qqqqff] (27.080558147599465,12.871849710542959)-- (10,70);
\draw [line width=1pt,dashed,color=qqqqff] (27.080558147599465,12.871849710542959)-- (20,50);
\draw [line width=1pt,dashed,color=qqqqff] (27.080558147599465,12.871849710542959)-- (40,30);
\draw [line width=1pt,dashed,color=qqqqff] (27.080558147599465,12.871849710542959)-- (90,10);
\draw [line width=1pt,dashed,color=qqffqq] (89.83150923646912,88.53724455912725)-- (70,95);
\draw [line width=1pt,dashed,color=qqffqq] (89.83150923646912,88.53724455912725)-- (40,70);
\draw [line width=1pt,dashed,color=qqffqq] (89.83150923646912,88.53724455912725)-- (60,70);
\draw [line width=1pt,dashed,color=qqffqq] (89.83150923646912,88.53724455912725)-- (95,60);
\draw [color=qqqqff](3.9007872968296593,8.898174707553851) node[anchor=north west] {$\mathbf{N_S}$};
\draw [color=qqffqq](95.46088215737039,101.12054873525942) node[anchor=north west] {$\mathbf{N_T}$};
\draw (87.18239256780974,95.5) node[anchor=north west] {$\mathbf{P_T}$};
\draw (20.457766475950947,15.024257003828726) node[anchor=north west] {$\mathbf{P_S}$};
\draw [line width=1pt,dashed,color=ffqqqq] (40,70)-- (60,50);
\draw [line width=1pt,dashed,color=ffqqqq] (60,50)-- (20,80);
\draw [line width=1pt,dashed,color=ffqqqq] (20,80)-- (90,10);
\draw [line width=1pt,dashed,color=qqqqff] (27.080558147599465,12.871849710542959)-- (20,80);
\draw [line width=1pt,dashed,color=qqqqff] (27.080558147599465,12.871849710542959)-- (60,50);
\draw [color=ffqqqq](68.5,53.8) node[anchor=north west] {$\mathbf{G_{\mathcal B}=(V_{\mathcal B}, E_{\mathcal B})}$};
\draw [color=qqffqq](68,90.19294247703937) node[anchor=north west] {$\mathbf{E_T}$};
\draw [color=qqqqff](28.736256065511594,33.56807368444457) node[anchor=north west] {$\mathbf{E_S}$};
\draw [line width=1pt,dashed,color=ffqqqq] (40,70)-- (95,60);
\draw [line width=1pt,dashed,color=ffqqqq] (40,70)-- (90,10);
\draw [line width=1pt,dashed,color=ffqqqq] (70,95)-- (40,30);
\begin{scriptsize}
\draw [color=ffqqqq] (20,80) circle (2.5pt);
\draw [color=ffqqqq] (40,30) circle (2.5pt);
\draw [color=ffqqqq] (70,95) circle (2.5pt);
\draw [color=ffqqqq] (40,70) circle (2.5pt);
\draw [color=ffqqqq] (95,60) circle (2.5pt);
\draw [color=ffqqqq] (60,70) circle (2.5pt);
\draw [color=ffqqqq] (60,50) circle (2.5pt);
\draw [color=ffqqqq] (90,10) circle (2.5pt);
\draw [color=ffqqqq] (10,70) circle (2.5pt);
\draw [color=ffqqqq] (20,50) circle (2.5pt);
\draw [fill=qqqqff] (27.080558147599465,12.871849710542959) circle (2.5pt);
\draw [fill=qqffqq] (89.83150923646912,88.53724455912725) circle (2.5pt);
\end{scriptsize}
\end{axis}
\end{tikzpicture}
\label{fig:graph2}
\end{minipage}
\end{figure}
%\pgfplotsset{compat=1.15}
\usetikzlibrary{arrows}
\definecolor{qqffqq}{rgb}{0,1,0}
\definecolor{qqqqff}{rgb}{0,0,1}
\definecolor{ffqqqq}{rgb}{1,0,0}
\begin{figure}[h!]
\centering
\begin{tikzpicture}[line cap=round,line join=round,>=triangle 45,x=1cm,y=1cm, scale = 0.5]
\begin{axis}[
x=0.1cm,y=0.1cm,
axis lines=middle,
xmin=-5,
xmax=105,
ymin=-5,
ymax=105,
xtick={-30,-20,...,160},
ytick={-30,-20,...,100},]
\clip(-30.537729395742623,-36.63351803502969) rectangle (160.8609499148995,109.73017790840248);
\draw [line width=1pt,color=ffqqqq] (20,80)-- (40,30);
\draw [line width=1pt,color=ffqqqq] (70,100)-- (40,70);
\draw [line width=1pt,color=ffqqqq] (100,60)-- (60,70);
\draw [line width=1pt,color=ffqqqq] (60,50)-- (90,10);
\draw [line width=1pt,color=ffqqqq] (10,70)-- (20,50);
\draw [line width=1pt,dashed,color=ffqqqq] (10,70)-- (20,80);
\draw [line width=1pt,dashed,color=ffqqqq] (20,50)-- (40,30);
\draw [line width=1pt,dashed,color=ffqqqq] (90,10)-- (40,30);
\draw [line width=1pt,dashed,color=ffqqqq] (20,80)-- (40,70);
\draw [line width=1pt,dashed,color=ffqqqq] (20,80)-- (70,100);
\draw [line width=1pt,dashed,color=ffqqqq] (40,70)-- (40,30);
\draw [line width=1pt,dashed,color=ffqqqq] (40,70)-- (60,70);
\draw [line width=1pt,dashed,color=ffqqqq] (70,100)-- (100,60);
\draw [line width=1pt,dashed,color=ffqqqq] (100,60)-- (90,10);
\draw [line width=1pt,dashed,color=ffqqqq] (60,50)-- (40,30);
\draw [line width=1pt,dashed,color=ffqqqq] (40,30)-- (60,70);
\draw [line width=1pt,dashed,color=ffqqqq] (60,70)-- (70,100);
\draw [line width=1pt,dashed,color=ffqqqq] (20,50)-- (20,80);
\draw [rotate around={0:(20,10)},line width=1pt,color=qqqqff,fill=qqqqff,fill opacity=0.25] (20,10) ellipse (1cm and 1cm);
\draw [rotate around={0:(90,90)},line width=1pt,color=qqffqq,fill=qqffqq,fill opacity=0.25] (90,90) ellipse (0.5cm and 0.5cm);
\draw [line width=1pt,dashed,color=ffqqqq] (10,70)-- (40,30);
\draw [line width=1pt,dashed,color=ffqqqq] (60,50)-- (60,70);
\draw [line width=1pt,dashed,color=ffqqqq] (60,50)-- (100,60);
\draw [line width=1pt,dashed,color=ffqqqq] (90,10)-- (60,70);
\draw [line width=1pt,dashed,color=qqqqff] (27.080558147599465,12.871849710542959)-- (10,70);
\draw [line width=1pt,dashed,color=qqqqff] (27.080558147599465,12.871849710542959)-- (20,50);
\draw [line width=1pt,dashed,color=qqqqff] (27.080558147599465,12.871849710542959)-- (40,30);
\draw [line width=1pt,dashed,color=qqqqff] (27.080558147599465,12.871849710542959)-- (90,10);
\draw [line width=1pt,dashed,color=qqffqq] (89.83150923646912,88.53724455912725)-- (70,100);
\draw [line width=1pt,dashed,color=qqffqq] (89.83150923646912,88.53724455912725)-- (40,70);
\draw [line width=1pt,dashed,color=qqffqq] (89.83150923646912,88.53724455912725)-- (60,70);
\draw [line width=1pt,dashed,color=qqffqq] (89.83150923646912,88.53724455912725)-- (100,60);
\draw [color=qqqqff](3.9007872968296593,8.898174707553851) node[anchor=north west] {$\mathbf{N_S}$};
\draw [color=qqffqq](95.46088215737039,101.12054873525942) node[anchor=north west] {$\mathbf{N_T}$};
\draw (87.18239256780974,96.48459456510545) node[anchor=north west] {$\mathbf{P_T}$};
\draw (20.457766475950947,15.024257003828726) node[anchor=north west] {$\mathbf{P_S}$};
\draw [line width=1pt,dashed,color=ffqqqq] (40,70)-- (60,50);
\draw [line width=1pt,dashed,color=ffqqqq] (60,50)-- (20,80);
\draw [line width=1pt,dashed,color=ffqqqq] (20,80)-- (90,10);
\draw [line width=1pt,dashed,color=qqqqff] (27.080558147599465,12.871849710542959)-- (20,80);
\draw [line width=1pt,dashed,color=qqqqff] (27.080558147599465,12.871849710542959)-- (60,50);
\draw [color=ffqqqq](72.11554151480937,53.76758828297254) node[anchor=north west] {$\mathbf{G_{\mathcal B}=(V_{\mathcal B}, E_{\mathcal B})}$};
\draw [color=qqffqq](69.13528526256754,90.19294247703937) node[anchor=north west] {$\mathbf{E_T}$};
\draw [color=qqqqff](28.736256065511594,33.56807368444457) node[anchor=north west] {$\mathbf{E_S}$};
\draw [line width=1pt,dashed,color=ffqqqq] (40,70)-- (100,60);
\draw [line width=1pt,dashed,color=ffqqqq] (40,70)-- (90,10);
\draw [line width=1pt,dashed,color=ffqqqq] (70,100)-- (40,30);
\begin{scriptsize}
\draw [color=ffqqqq] (20,80) circle (2.5pt);
\draw [color=ffqqqq] (40,30) circle (2.5pt);
\draw [color=ffqqqq] (70,100) circle (2.5pt);
\draw [color=ffqqqq] (40,70) circle (2.5pt);
\draw [color=ffqqqq] (100,60) circle (2.5pt);
\draw [color=ffqqqq] (60,70) circle (2.5pt);
\draw [color=ffqqqq] (60,50) circle (2.5pt);
\draw [color=ffqqqq] (90,10) circle (2.5pt);
\draw [color=ffqqqq] (10,70) circle (2.5pt);
\draw [color=ffqqqq] (20,50) circle (2.5pt);
\draw [fill=qqqqff] (27.080558147599465,12.871849710542959) circle (2.5pt);
\draw [fill=qqffqq] (89.83150923646912,88.53724455912725) circle (2.5pt);
\end{scriptsize}
\end{axis}
\end{tikzpicture}
\end{figure}


%\newcommand{\sign}{{\text{sign}}}


%\newcommand{\determinant}[3]{\det({#1#2#3})}



Since $\ES$ and $\ET$ are not fixed, the determinants in Remark \ref{rem:determinants} also depend on the location of $P_S$ and $P^{}_T$.  Hence, it is essential to model the previous constraint by using binary variables. We only focus on the case of $E_S$ but the same rationale is used for $E_T$.

Let $B\in\B$ be a barrier and $P_B^i$ an endpoint of $B$. Hence, the edge $(P^{}_S, P_B^i)$ belongs to $\ES$ if
$$\overline{P^{}_SP^i_B}\cap B''=\emptyset,\quad \forall B''\in\B,$$
or,
%Firstly, we will expose the constraints for two line segments $\segment{A}{B}$ and $\segment{C}{D}$ in general to model the previous remark. Afterwards, we will focus on the case for edges of $\ES$. 
by the preceding subsection, if

%Let $B\in\B$ a barrier and $P_B^i$ and endpoint of $B$. Hence, the edge $(P_S, P_B^i)$ belongs to $\ES$ if $\overline{P^{}_SP^i_B}\cap B'=\emptyset$, for all $B'\in\B$
%or equivalently, by the Remark \ref{rem:determinants}:
%\begin{equation*}
%\normalfont{\text{sign}}\left(\determinant{P_S^{}}{P_{B'}^1}{P_{B'}^2}\right) = \normalfont{\text{sign}}\left(\determinant{P_B^i}{P_{B'}^1}{P_{B'}^2}\right)
%\quad
%\text{or}
%\quad
%\normalfont{\text{sign}}\left(\determinant{P^{}_S}{P_{B'}^1}{P_{B'}^2}\right) = \normalfont{\text{sign}}\left(\determinant{P_B^i}{P_{B'}^1}{P_{B'}^2}\right),
%\quad\forall B'\in\B.
%\end{equation*}
%$$=\emptyset,\quad\forall B''\in\B,$$

%However, if those segments intersect, we need to ensure that $\deltacheck{P_S^{}}{P_B^i}{P_{B'}^1}{P_{B'}^2}$ is zero. It can be done by inserting the term $\epsilon \deltacheck{P_S^{}}{P_B^i}{P_{B'}^1}{P_{B'}^2}$ in the objective function, where $\epsilon>0$ is a small value.
\newcommand{\varepsilonvar}[2]{\varepsilon(#1#2)}
% At this point, we will focus on the description of the edges of $\ES$.
$$\deltacheck{P^{}_S}{P_B^i}{P^1_{B''}}{P^2_{B''}}=1,\quad\forall B''\in\B.$$
Hence, if we denote by $\varepsilonvar{P^{}_S}{P_B^i}$ the binary variable that is one when $(P^{}_S,P_B^i)\in\ES$ and zero otherwise, this variable can be represented by means of the following inequalities:
\begin{equation*}\tag{$\varepsilon$-C}\label{eq:varepsilonC}
\left[\sum_{B''\in\mathcal B}\deltacheck{P^{}_S}{P_B^i}{P^1_{B''}}{P^2_{B''}}-|\mathcal B|\right] + 1\leq \varepsilonvar{P^{}_S}{P_B^i}\leq \frac{1}{|\B|}\sum_{B''\in\mathcal B}\deltacheck{P^{}_S}{P_B^i}{P^1_{B''}}{P^2_{B''}}.
\end{equation*}
If there is a barrier $B'\in\B$ that intersects the segment $\overline{P^{}_SP_B^i}$, then $\deltacheck{P^{}_S}{P_B^i}{P^1_{B''}}{P^2_{B''}}$ is zero and the second inequality enforces $\varepsilonvar{P^{}_S}{P_B^i}$ to be zero because the right hand side is fractional and the first inequality is non-positive. Nonetheless, if there is no barrier that intersects the segment, then $\varepsilonvar{P^{}_S}{P_B^i}$ is equals to one, because the left hand side of the first inequality is one and the right hand side of the second inequality too.

\CV{Las variables epsilon sobre los segmentos no tiene la notacion de segmentos, ni las y, ni las g. Lo ponemos todo de la misma forma?}

\newcommand{\yvar}[2]{y(#1#2)}

Now, we can define the path that the drone can follow by taking into account the edges of the induced graph. Let $\yvar{PQ}$ be the binary variable that is one if the drone goes from $P$ to $Q$. Then, the inequalities
\begin{equation*}\tag{y-C}\label{eq:yC}
\yvar{P^{}_S}{P_B^i}\leq \varepsilonvar{P^{}_S}{P_B^i},\quad\forall P_B^i\in \VB,
\end{equation*}
assure that the drone can go from $P_S^{}\in \VS$ to a point of a barrier only if it does not cross any barrier. 


Now, we have the necessary elements to present our MINLP formulation for the \SPPN.

\begin{mini*}
{}{\sum_{(P,Q)\in E}\dvar{P}{Q}\yvar{P}{Q}}
{\label{eq:Example1}}{\tag{H-SPPN}}
\addConstraint{\sum_{\{Q:(P, Q)\in E\}}\yvar{P}{Q}-\sum_{\{Q:(Q, P)\in E\}}\yvar{Q}{P}}{=\left\{
\begin{array}{rl} 
1, & \text{if } P\in\VS, \\
0, & \text{if } P\in\VB, \\
-1, & \text{if }P\in\VT.
\end{array}
\right.}
\addConstraint{\eqref{eq:alphaC},\eqref{eq:betaC},\eqref{eq:deltaC},\eqref{eq:gammaC},\eqref{eq:varepsilonC},\eqref{eq:yC}}{ }
\addConstraint{\eqref{eq:dC}, \eqref{eq:nC}.}{ }
\end{mini*}

The objective function minimizes the length of the path followed by the drone on the edges of the induced graph $G$. The first constraints are the flow conservation constraints, the second constraints represent the sets $\ES$ and $\ET$ and the third ones state that the points selected must be in their respective neighborhoods.

\pgfplotsset{compat=1.15}
\usetikzlibrary{arrows}
\definecolor{ududff}{rgb}{0.30196078431372547,0.30196078431372547,1}
\definecolor{qqffqq}{rgb}{0,1,0}
\definecolor{qqqqff}{rgb}{0,0,1}
\definecolor{ffqqqq}{rgb}{1,0,0}
\begin{figure}[h!]
\caption{Optimal solution for this instance of the \SPPN}
\centering
\begin{tikzpicture}[line cap=round,line join=round,>=triangle 45,x=1cm,y=1cm, scale=0.65]
\begin{axis}[
x=0.1cm,y=0.1cm,
axis lines=middle,
xmin=-5,
xmax=105,
ymin=-5,
ymax=105,
xtick={-30,-20,...,160},
ytick={-30,-20,...,95},]
\draw [line width=1pt,color=ffqqqq] (20,80)-- (40,30);
\draw [line width=1pt,color=ffqqqq] (70,95)-- (40,70);
\draw [line width=1pt,color=ffqqqq] (95,60)-- (60,70);
\draw [line width=1pt,color=ffqqqq] (60,50)-- (90,10);
\draw [line width=1pt,color=ffqqqq] (10,70)-- (20,50);
\draw [rotate around={0:(20,10)},line width=1pt,color=qqqqff,fill=qqqqff,fill opacity=0.25] (20,10) ellipse (1cm and 1cm);
\draw [rotate around={0:(90,90)},line width=1pt,color=qqffqq,fill=qqffqq,fill opacity=0.25] (90,90) ellipse (0.5cm and 0.5cm);
\draw [->,line width=1pt] (27.07,17.07) -- (40,30);
\draw [->,line width=1pt] (40,30) -- (60,70);
\draw [->,line width=1pt] (60,70) -- (85.36,88.14);
\draw (23.901825857302796,25.150339300103585) node[anchor=north west] {$\mathbf{P_S}$};
\draw (76.40719360287545,93.82231539794059) node[anchor=north west] {$\mathbf{P_T}$};
\begin{scriptsize}
\draw [color=ffqqqq] (20,80) circle (2.5pt);
\draw [color=ffqqqq] (40,30) circle (2.5pt);
\draw [color=ffqqqq] (70,95) circle (2.5pt);
\draw [color=ffqqqq] (40,70) circle (2.5pt);
\draw [color=ffqqqq] (95,60) circle (2.5pt);
\draw [color=ffqqqq] (60,70) circle (2.5pt);
\draw [color=ffqqqq] (60,50) circle (2.5pt);
\draw [color=ffqqqq] (90,10) circle (2.5pt);
\draw [color=ffqqqq] (10,70) circle (2.5pt);
\draw [color=ffqqqq] (20,50) circle (2.5pt);
\draw [fill=ududff] (27.07,17.07) circle (2.5pt);
\draw [fill=qqffqq] (85.36,88.14) circle (2.5pt);
\end{scriptsize}
\end{axis}
\end{tikzpicture}
\label{fig:solution_spp}
\end{figure}

\subsubsection{Reformulating the \SPPN}\label{section:reformulation}
The formulation for the \SPPN \ presented above can be simplified by taking into account the following observation.
\begin{prop}
There exists a finite dominant set, $N_S^*$, of possible candidates to be in  $N_S$. Moreover,
$$N_S^*=\{P_S(P_B^i)=\argmin_{P\in N_S} \|P_S - P_B^i\|:(P, P_B^i)\in E_S\}.$$
\end{prop}
\begin{proof}
Note that the point selected in $N_S$ in an optimal solution for \SPPN \ must the one that gives the minimum distance to the point of the first visited barrier in the optimal solution. Therefore, $N^*$ must be composed, at most, by points in the set $\{P_S(P_B^i)=\argmin_{P\in N_S} \|P_S - P_B^i\|:(P, P_B^i)\in E_S\}$.
\end{proof}
Therefore, we can compute the set $N_S^*$ by solving a convex problem for each endpoint of the barriers:
$$N_S^*=\{P_S(P_B^i)=\argmin_{P_S\in N_S} \|P_S - P_B^i\|:(P_S, P_B^i)\in E_S\}.$$
Moreover, the point chosen in the solution, $P_S$ ,can be represented by the points in $N^*$ as shown below:
\begin{align*}\tag{N$^*$-C}\label{eq:nCs}
P_S&=\sum_{P_B^i\in V_{\mathcal B}}\mu_S(P_B^i)P_S(P_B^i),\\
1&=\sum_{P_B^i\in V_{\mathcal B}}\mu_S(P_B^i),
\end{align*}
where $\mu_S(P_B^i)$ is a binary variable that assumes the value one if $P_S(P_B^i)$ is selected to go from from $N_S$ to the first barrier.
The major advantage of this approach is that the whole graph $G$ is fixed and the incident edges can be computed for each point $P_S(P_B^i)$, $P_B^i\in V_{\mathcal B}$ separately. Defining again the variable $y^*$ for the edges in $E$, the new formulation for the \SPPN \ can be expressed as the following simplified program:
\begin{mini*}
{}{\sum_{(P,Q)\in E}\dvar{P}{Q}\yvar{P}{Q}}
{\label{eq:Example2}}{\tag{H-SPPN$^*$}}
\addConstraint{\sum_{\{Q:(P, Q)\in E\}}\yvar{P}{Q}-\sum_{\{Q:(Q, P)\in E\}}\yvar{Q}{P}}{=\left\{
\begin{array}{rl} 
1, & \text{if } P\in\VS, \\
0, & \text{if } P\in\VB, \\
-1, & \text{if }P\in\VT.
\end{array}
\right.}
\addConstraint{\eqref{eq:dC}, \eqref{eq:nCs}.}{ }
\end{mini*}

\CV{
\begin{prop}
The \SPPN \ can be solved in polynomial time.
\end{prop}


The proof follows using the finite dominant sets $N^*_{S}$ and $N^*_{T}$ and taking the minimum of the lengths of the solutions for the shortest path problem that can be obtained for each pair of points.
}

\subsection{A formulation for the \TSPN}
\JP{To present our formulation for the \TSPN, the graph induced by the endpoints of the barriers and the neighborhoods is different from the previous one for the \SPPN. For its description, we introduce the following sets:}

\textcolor{red}{Discutir si abrir otra nueva sección con el modelo en el que las bolas se ven o añadir un comentario en esta subsección modificando solo la descripción de $\EN$}

\begin{itemize}
\item $\VN=\{P_N:N\in\mathcal N\}$: set of the points selected in the neighborhoods $\mathcal N$ to be visited.
\item $\VB=\{P^1_B, P^2_B:B=\overline{P^1_B P^2_B}\in \mathcal B\}$: set of vertices that form the barriers of the problem.
\item $\EN=\{(P_N, P^i_{B}):P^i_B\in V_\B\text{ and } \overline{P_NP^i_B}\cap B''=\emptyset,\forall B''\in\B,\:i=1,2\}$: set of edges formed by the line segments that join the point selected in the neighborhoods of $\mathcal N$ and every endpoint in the barriers and do not cross any barrier in $\B$.
\item $\EB=\{(P^{i}_B, P^{j}_{B'}):P^i_B, P^j_{B'}\in \VB \text{ and } \overline{P^i_B P^j_{B'}}\cap B''=\emptyset,\:\forall B''\in\mathcal B,\:i, j=1,2\}$: set of edges formed by the line segments that join two vertices of $V_{\mathcal B}$ and do not cross any barrier in $\B$.
\end{itemize} 

\JP{Following the same idea as before, we set $G=(V,E)$ induced by the barriers and the neighborhoords, where $V=\VN\cup\VB$ and $E=\EN\cup\EB$. 

The rationale of the formulation for the \TSPN is to consider the variant called Steiner TSP (STSP) (reference to this problem), where some nodes in $\VB$ do not have to be visited, but if necessary they can be visited more than once. 

It is well known that it is possible to convert any instance of the STSP into an instance of the standard TSP, by computing the shortest paths between every pair of required nodes, when these nodes are fixed. However, in our problem, since the points in the neighborhoods are not fixed, this simplification cannot be applied to obtain the optimal solution for the \TSPN, although it may produce an approximation to generate lower bounds for the \TSPN.}


\newcommand{\gvar}[2]{g(#1#2)}

\JP{Our formulation rest on adjusting single-commodity flow formulationto ensure connectivity. We can assume wlog that the neighborhood $N_1$ is required and the drone departs from that depot (assuming to be $N_1$) with $|\mathcal N|-1$ units of commodity. The idea is that the model must deliver one unit of commodity to each one of the required neighborhoods. Then, for each edge $(P, Q)\in E$, we define the following variables:}
\begin{itemize}
	\item $\yvar{P}{Q}$, binary variable that is equals to one if the drone goes from $P$ to $Q$.
	\item $\gvar{P}{Q}$, non-negative continuous variable that represents the amount of the commodity passing through the edge $(P, Q)$.
\end{itemize}

Hence, we can adjust the single-commodity flow formulation to the induced graph $G$ as follows:

\begin{mini*}
{}{\sum_{(P,Q)\in E}\dvar{P}{Q}\yvar{P}{Q}}
{\label{eq:Example1}}{\tag{H-TSPN}}
\addConstraint{\sum_{\{Q:(P_N, Q)\in \EN\}}\yvar{P_N}{Q}}{\geq 1,}{\quad\forall P_N\in \VN}
\addConstraint{\sum_{\{Q:(P, Q)\in E\}}\yvar{P}{Q}}{= \sum_{\{Q:(Q, P)\in E\}}\yvar{Q}{P},}{\quad\forall P\in V}
\addConstraint{\sum_{\{Q:(Q, P_N)\in \EN\}}\gvar{Q}{P_N}-\sum_{\{Q:(P_N, Q)\in \EN\}}\gvar{P_N}{Q}}{= 1,}{\quad\forall P_N\in \VN\setminus\{P_{N_1}\}}
\addConstraint{\sum_{\{Q:(Q, P)\in E\}}\gvar{Q}{P}-\sum_{\{Q:(P, Q)\in E\}}\gvar{P}{Q}}{= 0,}{\quad\forall P\in \VB}
\addConstraint{\gvar{P}{Q}}{\leq (|\mathcal N|-1)\yvar{P}{Q},}{\quad\forall (P,Q)\in E}
\addConstraint{\eqref{eq:alphaC},\eqref{eq:betaC},\eqref{eq:deltaC},\eqref{eq:gammaC},\eqref{eq:varepsilonC},\eqref{eq:yC}}{ }
\addConstraint{\eqref{eq:dC}, \eqref{eq:nC}.}{ }
\end{mini*}

\JP{The first constraints impose that the drone departs from each neighborhood. The second constraints are the flow conservation constraints. The third inequalities ensure that one unit of commodity is delivered to each required neighborhood. The fourth ones ensure that the amount of commodity passing through the points in the barriers is not consumed. Finally, the last inequalities enforce that if some commodity passes along an edge only if this edge is used in the tour.}

\pgfplotsset{compat=1.15}
\usetikzlibrary{arrows}
\definecolor{bfffqq}{rgb}{0.7490196078431373,1,0}
\definecolor{ffxfqq}{rgb}{1,0.4980392156862745,0}
\definecolor{qqffqq}{rgb}{0,1,0}
\definecolor{qqqqff}{rgb}{0,0,1}
\definecolor{ffqqqq}{rgb}{1,0,0}
\begin{figure}[h!]
\caption{Solution for the instance of the \TSPN}
\centering
\begin{tikzpicture}[line cap=round,line join=round,>=triangle 45,x=1cm,y=1cm, scale=0.65]
\begin{axis}[
x=0.1cm,y=0.1cm,
axis lines=middle,
xmin=-5,
xmax=105,
ymin=-5,
ymax=105,
xtick={-30,-20,...,160},
ytick={-30,-20,...,95},]
\clip(-30.537729395742623,-29.679586779798765) rectangle (175.76223117610863,116.68410916363341);
\draw [line width=1pt,color=ffqqqq] (20,80)-- (40,30);
\draw [line width=1pt,color=ffqqqq] (70,95)-- (40,70);
\draw [line width=1pt,color=ffqqqq] (95,60)-- (60,70);
\draw [line width=1pt,color=ffqqqq] (60,50)-- (90,10);
\draw [line width=1pt,color=ffqqqq] (10,70)-- (20,50);
\draw [rotate around={0:(20,10)},line width=1pt,color=qqqqff,fill=qqqqff,fill opacity=0.25] (20,10) ellipse (1cm and 1cm);
\draw [rotate around={0:(90,90)},line width=1pt,color=qqffqq,fill=qqffqq,fill opacity=0.25] (90,90) ellipse (0.5cm and 0.5cm);
\draw [rotate around={0:(35,85)},line width=1pt,color=ffxfqq,fill=ffxfqq,fill opacity=0.25] (35,85) ellipse (0.9cm and 0.9cm);
\draw [rotate around={0:(85,40)},line width=1pt,color=bfffqq,fill=bfffqq,fill opacity=0.25] (85,40) ellipse (1.1cm and 1.1cm);
\draw [->,line width=1pt] (27.07,17.07) -- (40,30);
\draw [->,line width=1pt] (40,30) -- (27.07,17.07);
\draw [->,line width=1pt] (40,30) -- (40,70);
\draw [->,line width=1pt] (40,70) -- (42.19,79.59);
\draw [->,line width=1pt] (42.19,79.59) -- (70,95);
\draw [->,line width=1pt] (70,95) -- (86.2,86.75);
\draw [->,line width=1pt] (86.2,86.75) -- (95,60);
\draw [->,line width=1pt] (95,60) -- (81.55,50.45);
\draw [->,line width=1pt] (81.55,50.45) -- (60,50);
\draw [->,line width=1pt] (60,50) -- (40,30);
\draw (20.94789460207184,18.16350179860903) node[anchor=north west] {$\mathbf{P_1}$};
\draw (86.35454360885365,93.50433831286361) node[anchor=north west] {$\mathbf{P_2}$};
\draw (36.349175863281,86.08483726584145) node[anchor=north west] {$\mathbf{P_3}$};
\draw (81.55301964690848,50.8774601568516) node[anchor=north west] {$\mathbf{P_4}$};
\begin{scriptsize}
\draw [color=ffqqqq] (20,80) circle (2.5pt);
\draw [color=ffqqqq] (40,30) circle (2.5pt);
\draw [color=ffqqqq] (70,95) circle (2.5pt);
\draw [color=ffqqqq] (40,70) circle (2.5pt);
\draw [color=ffqqqq] (95,60) circle (2.5pt);
\draw [color=ffqqqq] (60,70) circle (2.5pt);
\draw [color=ffqqqq] (60,50) circle (2.5pt);
\draw [color=ffqqqq] (90,10) circle (2.5pt);
\draw [color=ffqqqq] (10,70) circle (2.5pt);
\draw [color=ffqqqq] (20,50) circle (2.5pt);
\draw [fill=qqqqff] (27.07,17.07) circle (2.5pt);
\draw [fill=qqffqq] (86.2,86.75) circle (2.5pt);
\draw [fill=ffxfqq] (42.19,79.59) circle (2.5pt);
\draw [fill=bfffqq] (81.55,50.45) circle (2.5pt);
\end{scriptsize}
\end{axis}
\end{tikzpicture}
\label{fig:solution_tspp}
\end{figure}

\CV{
\begin{prop}
The \TSPN \ is NP-complete.
\end{prop}

Note that, once a point is fixed in each neighborhood, the problem obtained in the induced graph $G$ is the Steiner TSP (STSP), that is NP-complete.

Finite dominating set.
}

\CV{
\subsection{Relaxing the assumptions of the problem: The \TSPVN}

In this subsection, we expose the differences that appear when, in the model of the \TSPN, we do not require that the barriers separate the neighborhoods completely, i.e., when going from one neighborhood to another one is possible without crossing any barrier. The main difference lies in the description of the edges of the graph induced by the neighborhoods and the endpoints of the barriers.

By taking the same approach, the sets that describe the graph in that case are:

\begin{itemize}
\item $\VN=\{P_N:N\in\mathcal N\}$: set of points in the neighborhoods $\mathcal N$ that must be visited.
\item $\VB=\{P^1_B, P^2_B:B=\overline{P^1_B P^2_B}\in \mathcal B\}$: set of vertices that form the barriers of the problem.
\item $V = \VN \cup \VB$.
\item $E=\{(P, P'):P, P' \in V \text{ and } \overline{PP'}\cap B''=\emptyset,\forall B''\in\B\}$: set of edges formed by the line segments that join every pair of points in $V$ that do not cross any barrier.
\end{itemize} 

The difference between the set of edges in the \TSPVN with respect to the graph in \TSPN is that, in the former case, the edges that join each pair of neighborhoods are considered. This fact leads to include product of continuous variables in the $\alpha$ constraints of the model that represent the determinants that determine if the two variable points in the neighborhoods cross any barrier or not and the problem becomes nonconvex.

}
%\textcolor{red}{¿Solucion del problema del TSP propuesto en el dibujo? }

%Note that this problem is not convex because the determinants that model the orientation of the points to check if there exist crossings produce product of variables. 

%\begin{equation*}
%\normalfont{\text{sign}}\left(\determinant{P_S^{}}{P_{B'}^1}{P_{B'}^2}\right) = \normalfont{\text{sign}}\left(\determinant{P_B^i}{P_{B'}^1}{P_{B'}^2}\right)
%\quad
%\text{or}
%\quad
%\normalfont{\text{sign}}\left(\determinant{P^{}_S}{P_{B'}^1}{P_{B'}^2}\right) = \normalfont{\text{sign}}\left(\determinant{P_B^i}{P_{B'}^1}{P_{B'}^2}\right),
%\quad\forall B'\in\B.
%\end{equation*}
%%$$=\emptyset,\quad\forall B''\in\B,$$
%\newcommand{\LS}[3]{L(#1|#2#3)}
%\newcommand{\US}[3]{U(#1|#2#3)}
%\newcommand{\alpha}[3]{\alpha^{+}(#1|#2#3)}
%\newcommand{\alphamenos}[3]{\alpha^{-}(#1|#2#3)}
%\newcommand{\alphacero}[3]{\alpha^{0\,}(#1|#2#3)}
%\textcolor{red}{Consideramos el caso en el que los puntos estan alineados?}
%
%To model the sign of the determinant $\determinant{P_S^{}}{P_{B'}^1}{P_{B'}^2}$, we define the following binary variables:
%\begin{itemize}
%\item $\alpha{P_S^{}}{P_{B'}^1}{P_{B'}^2}$, that is one if $\determinant{P_S^{}}{P_{B'}^1}{P_{B'}^2}$ is strictly positive and zero, otherwise.
%\item $\alphamenos{P_S^{}}{P_{B'}^1}{P_{B'}^2}$, that is one if $\determinant{P_S^{}}{P_{B'}^1}{P_{B'}^2}$ is strictly negative and zero, otherwise.
%\item $\alphacero{P_S^{}}{P_{B'}^1}{P_{B'}^2}$, that is one if $\determinant{P_S^{}}{P_{B'}^1}{P_{B'}^2}$ is zero and zero, otherwise.
%\end{itemize}
%
%
%It is possible to represent the sign condition by including the following constraints:
%%\begin{equation}\tag{sign+}\label{eq:sign}
%%\LS{P_S^{}}{P_{B'}^1}{P_{B'}^2}\left(1-\alpha{P_S^{}}{P_{B'}^1}{P_{B'}^2}\right)\leq \determinant{P_S^{}}{P_{B'}^1}{P_{B'}^2}\leq \US{P_S^{}}{P_{B'}^1}{P_{B'}^2}\alpha{P_S^{}}{P_{B'}^1}{P_{B'}^2},
%%\end{equation}
%\begin{align*}\tag{sign}\label{eq:sign}
%\determinant{P_S^{}}{P_{B'}^1}{P_{B'}^2}&\leq \US{P_S^{}}{P_{B'}^1}{P_{B'}^2}\:\alpha{P_S^{}}{P_{B'}^1}{P_{B'}^2},\\
%\determinant{P_S^{}}{P_{B'}^1}{P_{B'}^2}&\geq \LS{P_S^{}}{P_{B'}^1}{P_{B'}^2}\:\alphamenos{P_S^{}}{P_{B'}^1}{P_{B'}^2},\\
%\alpha{P_S^{}}{P_{B'}^1}{P_{B'}^2}+\alphamenos{P_S^{}}{P_{B'}^1}{P_{B'}^2}+\alphacero{P_S^{}}{P_{B'}^1}{P_{B'}^2}& = 1,
%\end{align*}
%where $\LS{P_S^{}}{P_{B'}^1}{P_{B'}^2}$ and $\US{P_S^{}}{P_{B'}^1}{P_{B'}^2}$ are a lower and a upper bound for the determinant, respectively. If the determinant is strictly positive, then $\alpha{P_S^{}}{P_{B'}^1}{P_{B'}^2}$ must be one to make the first inequality feasible, the second inequality is always fulfilled and the third one sets the other indicator variables to zero. The analagous case happens if the determinant is strictly negative. However, if the determinant is equal to zero, both inequalities are fulfilled and it is necessary to include the term $\epsilon (1-\alphacero{P_S^{}}{P_{B'}^1}{P_{B'}^2})$ in the objective function, where $\epsilon>0$ is a small value, to ensure that $\alphacero{P_S^{}}{P_{B'}^1}{P_{B'}^2}$ is equal to one when the determinant is zero.
%
%\newcommand{\betamas}[4]{\beta^{+}(#1#2|#3#4)}
%\newcommand{\betamenos}[4]{\beta^{-}(#1#2|#3#4)}
%\newcommand{\betacero}[4]{\beta^{0\,}(#1#2|#3#4)}
%
%At this point, it is required to introduce the binary variables that checks if both determinants are positive, negative or null:
%
%\begin{itemize}
%\item $\betamas{P_S^{}}{P_B^i}{P_{B'}^1}{P_{B'}^2}$, that is one if $\determinant{P_S^{}}{P_{B'}^1}{P_{B'}^2}$ and $\determinant{P_B^i}{P_{B'}^1}{P_{B'}^2}$ are both positive.
%\item $\betamenos{P_S^{}}{P_B^i}{P_{B'}^1}{P_{B'}^2}$, that is one if $\determinant{P_S^{}}{P_{B'}^1}{P_{B'}^2}$ and $\determinant{P_B^i}{P_{B'}^1}{P_{B'}^2}$ are both negative.
%\item $\betacero{P_S^{}}{P_B^i}{P_{B'}^1}{P_{B'}^2}$, that is one if $\determinant{P_S^{}}{P_{B'}^1}{P_{B'}^2}$ and $\determinant{P_B^i}{P_{B'}^1}{P_{B'}^2}$ are both zero.
%\end{itemize}
%
%Hence, the $\beta$ variables can be represented by the product of the $\alpha$ variables:
%\begin{align*}
%\betamas{P_S^{}}{P_B^i}{P_{B'}^1}{P_{B'}^2}&=\alpha{P_{B'}^1}{P^{}_S}{P^i_B}\:\alpha{P_{B'}^2}{P^{}_S}{P^i_B},\\
%\betamenos{P_S^{}}{P_B^i}{P_{B'}^1}{P_{B'}^2}&=\alphamenos{P_{B'}^1}{P^{}_S}{P^i_B}\:\alphamenos{P_{B'}^2}{P^{}_S}{P^i_B},\\
%\betacero{P_S^{}}{P_B^i}{P_{B'}^1}{P_{B'}^2}&=\alphacero{P_{B'}^1}{P^{}_S}{P^i_B}\:\alphacero{P_{B'}^2}{P^{}_S}{P^i_B}.
%\end{align*}
%
%These products can be linearized by means of these constraints:
%$$
%\left\{\begin{array}{rcl}
%\betamas{P_S^{}}{P_B^i}{P_{B'}^1}{P_{B'}^2} & \leq & \alpha{P_{B'}^1}{P^{}_S}{P^i_B},\\\\
%\betamas{P_S^{}}{P_B^i}{P_{B'}^1}{P_{B'}^2} & \leq & \alpha{P_{B'}^2}{P^{}_S}{P^i_B},\\\\
%\betamas{P_S^{}}{P_B^i}{P_{B'}^1}{P_{B'}^2} & \geq & \alpha{P_{B'}^1}{P^{}_S}{P^i_B} + \alpha{P_{B'}^2}{P^{}_S}{P^i_B} - 1.\end{array}\right.
%$$
%
% $\alpha{P_S^{}}{P_{B'}^1}{P_{B'}^2}$ that attains minus one if the sign of the determinant is negative, zero, if the determinant is zero, and one, if the determinant is positive. The following constraints set the sign condition:
%\begin{equation}\tag{sign}\label{eq:sign}
%\LS{P_S^{}}{P_{B'}^1}{P_{B'}^2}\:\alpha{P_S^{}}{P_{B'}^1}{P_{B'}^2}\leq \determinant{P_S^{}}{P_{B'}^1}{P_{B'}^2}\leq \US{P_S^{}}{P_{B'}^1}{P_{B'}^2})\left(1-\alpha{P_S^{}}{P_{B'}^1}{P_{B'}^2}\right),
%\end{equation}
% If the determinant is negative, then the right side of the equation must be zero, and $\alpha{P_S^{}}{P_{B'}^1}{P_{B'}^2}$ is one. Analogously, if the determinant is non-negative, then the left side must be zero and $\alpha{P_S^{}}{P_{B'}^1}{P_{B'}^2}$ is also zero.
%
%%In the left picture, the number of incident edges $|E_S|$ on $N_S$ is 4 whereas the right picture is 6.

\section{Strengthening the formulations}\label{section:strengthening}
\textcolor{red}{Incluir el resultado estructural que da una cota superior del numero de bolas que se puede generar}
\subsection{Preprocessing}\label{subsection:preprocessing}
In this subsection, we show a preprocessing result that allows to fix some variables by taking into account the relative position between the neighborhoods and the barriers.
In particular, we are going to present an outcome that ensures that there are some barriers whose endpoints can not be incident in the edges of $\EN$ and it is not necessary to include it in $\EN$. 

Let denote

\begin{align*}
\text{cone}(P,Q,R)&:=\{\mu_1 \overrightarrow{PQ}+\mu_2\overrightarrow{PR}:\mu_1,\mu_2\geq 0\},\\
\text{cone}(P,Q,R)^-&:=\{\mu_1 \overrightarrow{PQ}+\mu_2\overrightarrow{PR}:\mu_1,\mu_2\geq 0,\:\mu_1+\mu_2\leq 1\},\\
\text{cone}(P,Q,R)^+&:=\{\mu_1 \overrightarrow{PQ}+\mu_2\overrightarrow{PR}:\mu_1,\mu_2\geq 0,\:\mu_1+\mu_2\geq 1\}.
\end{align*}
Note that $cone(P, Q, R)$ is the union of $\text{cone}(P,Q,R)^-$ and $\text{cone}(P,Q,R)^+$. It is also important to remark that $\text{cone}(P,Q,R)^+$ is the part of cone that crosses the barrier $\overline{QR}$ when we consider a segment whose endpoints are $P$ and another point of this set, i.e.
$$\text{cone}(P,Q,R)^+=\{P':\overline{PP'}\cap\overline{QR}\neq\emptyset\}.$$

Let $B = \overline{P^1_BP^2_B}\in\mathcal B$ a barrier. In the following proposition, we give a sufficient condition to not include the edge $(P^{}_N, P^i_B)$ in $\EN$:

\begin{prop}

Let $B'=\overline{P^1_{B'}P^2_{B'}}\in\mathcal B$ and $\normalfont{\text{cone}}(P^i_B, P^1_{B'}, P^2_{B'})^+$ the conical hull generated by these points. If
$$N\subset\bigcup_{B'\in\mathcal B}\normalfont{\text{cone}}(P^i_B, P^1_{B'}, P^2_{B'})^+,$$
then $(P^{}_N, P^i_B)\not\in \EN$.

%Let $B'=\overline{P^1_{B'}P^2_{B'}}\in\mathcal B$ and $Q(B')$ the intersection point of the straight lines that join points of $N$ and $B'$ that lies outside of the convex hull generated by thse line segments. Let $\normalfont{\text{cone}}(\{\overrightarrow{Q(B')P^1_N},\overrightarrow{Q(B')P^2_N}\})$ the conical hull of these vectors. If
%$$B\subset\bigcup_{B'\in\mathcal B}\normalfont{\text{cone}}(\{\overrightarrow{Q(B')P^1_N},\overrightarrow{Q(B')P^2_N}\}),$$
%then $(P^{}_N, P^i_{B})\not\in \EN$, $i=1,2$.
\end{prop}
\begin{proof}
If $P_N\in N$, then there exists a $B'\in\mathcal B$ such that 
$P_N\in \normalfont{\text{cone}}(P^i_B, P^1_{B'}, P^2_{B'})^+$. Therefore, $\overline{P^i_B P^{}_N}\cap B'\neq\emptyset$ and $(P^{}_N, P^i_B)\not\in \EN$, as we claimed.
\end{proof}

We can check computationally the condition of the previous proposition by using the following procedure. The first issue that is solved is the one when the neighborhoods are segments. Let $N = \overline{P^1_N P^2_N}$ be a line segment and $r_N$ the straight line that contains the line segment $N$ that is represented as:

$$r_N:P_N^1+\lambda\overrightarrow{P_N^1P_N^2},\qquad\lambda\in\mathbb R.$$


\begin{algorithm}[H]
\caption{Checking computationally if $(P^{}_N, P^i_B)\not\in \EN$ when $N$ is a segment.}
\label{alg:Algorithm1}
\SetKwInOut{Input}{Initialization}\SetKwInOut{Output}{output}

 \Input{Let $P^i_B$ be the point whose edge $(P^i_B, P^{}_N)$ is going to check if $(P^{}_N, P^i_B)\not\in \EN$.\\
Set $points = \{P^1_N, P^2_N\}$, $lambdas=\{0, 1\}$.}
% Set $LB=0$, $UB=+\infty$, $\bar z=z^0$.
\For{$B''\in\mathcal B$}{
	\For{$j\in\{1, 2\}$}{
		Compute the straight line 
		$$r(P^i_B, P^j_{B''}) = P^i_B + \mu^j_{B''}\overrightarrow{P^i_BP^j_{B''}},$$ 
		that contains the points $P^i_B$ and $P^j_{B''}$. \\
	
		Intersect $r(P^i_B, P^j_{B''})$ and $r_N$ in the point $Q^j_{B''}$ and compute $			\overline{\mu}^j_{B''}$ such that 
		$$Q^j_{B''}=P_B^i + \overline{\mu}^j_{B''} \overrightarrow{P^i_BP^j_{B''}}.$$\\
	
		\If{$|\overline{\mu}^j_{B''}|\geq 1$}{
			Compute $\lambda^j_{B''}$ such that 
			$$Q^j_{B''}=P^1_N+\lambda^j_{B''}\overrightarrow{P^1_NP^2_N}.$$\\
			\If{$\overline{\mu}^j_{B''}\geq 1$}{
				Include $\lambda^j_{B''}$ in $lambdas$.}
			\Else{
				\If{$\lambda^j_{B''}\geq 0$}{
					Set $\lambda^j_{B''}=M<<0$ and include it in $lambdas$.}
				\Else{
					Set $\lambda^j_{B''}=M>>0$ and include it in $lambdas$.}
	   			}	
			}
		}
	}
Order in non-decreasing order the set $lambdas$.\\
If it is verified that
\begin{align*}
\min\{\lambda_{B'}^1, \lambda_{B'}^2\}\leq 0\leq\max\{\lambda_{B'}^1, \lambda_{B'}^2\},&\quad\text{ for some } B'\in\mathcal B,\\
\min\{\lambda_{B'}^1, \lambda_{B'}^2\}\leq 1\leq\max\{\lambda_{B'}^1, \lambda_{B'}^2\},&\quad\text{ for some } B'\in\mathcal B,\\
\min\{\lambda_{B'}^1, \lambda_{B'}^2\}\leq\lambda_{B''}^j\leq\max\{\lambda_{B'}^1, \lambda_{B'}^2\},&\quad\text{ for some } B'\in\mathcal B\setminus\{B''\},\quad\forall \lambda_{B''}^j\in lambdas\setminus\{M\},
\end{align*}
or
$$
\min\{\lambda_{B'}^1, \lambda_{B'}^2\}\leq 0, 1\leq\max\{\lambda_{B'}^1, \lambda_{B'}^2\},\quad\text{ for some } B'\in\mathcal B,
$$
then $(P^{}_N, P^i_B)\not\in \EN$.

\end{algorithm}

\input{figures/Preprocessing - First case}

Note that this algorithm also allows us to decide if the drone can access to a point of a barrier from any point of the neighborhood $N$. It is enough to check in (15) that 
$$0\not\in \left[\min\{\lambda_{B'}^1, \lambda_{B'}^2\},\max\{\lambda_{B'}^1, \lambda_{B'}^2\}\right]\quad\text{and}\quad 1\not\in \left[\min\{\lambda_{B'}^1, \lambda_{B'}^2\},\max\{\lambda_{B'}^1, \lambda_{B'}^2\}\right],\quad\forall B'\in\mathcal B.$$

For the case when $N$ is an ellipse, the same rationale can be followed. The idea is to generate the largest line segment contained in the ellipse and repeat the procedure exposed in the Algorithm \ref{alg:Algorithm1}. Let $F_1$ and $F_2$ the focal points of $N$.

\begin{algorithm}[H]
\caption{Checking computationally if $(P^{}_N, P^i_B)\not\in \EN$ when $N$ is an ellipse.}
\label{alg:Algorithm2}
\SetKwInOut{Input}{Initialization}\SetKwInOut{Output}{output}

 \Input{Let $P^i_B$ be the point whose edge $(P^i_B, P^{}_N)$ is going to check if $(P^{}_N, P^i_B)\not\in \EN$.\\
Set $points = \{\}$, $lambdas=\{\}$.}
% Set $LB=0$, $UB=+\infty$, $\bar z=z^0$.}
Compute the straight line $r(F^1, F^2)$.\\
Intersect $r(F^1, F^2)$ and $\partial N$ in the points $P_N^1$ and $P_N^2$. \\
Include $P_N^1$ and $P_N^2$ in $points$. \\
Apply the Algorithm \ref{alg:Algorithm1}.
\end{algorithm}


\textcolor{red}{Estudiar resultados que eliminen algunas de las posibles aristas de $E_S$ y $E_T$}
\textcolor{red}{Hablar que las aristas $E_{\mathcal B}$ se pueden preprocesar porque los puntos están fijados}

\CV{Es posible encontrar un caso en el que utilizar la barrera como arista sea lo mejor?}

\subsection{Valid inequalities}
This subsection is devoted to show some results that adjust the bigM constants that appear in the previous formulation, specifically, in the \eqref{eq:alphaC}, where the modelling of the sign requires to compute the lower and upper bounds $L$ and $U$, respectively. We are going to determine these bounds explicitly for the cases when the neighborhoods are ellipses and segments.

Let $\overline{P^1_{B'}P^2_{B'}}=B'\in\B$ be a barrier and $P_N\in N$. Let $\determinant{P_N}{P_{B'}^1}{P_{B'}^2}$ also be the determinant whose value must be bounded. Hence, the solution of the following problem gives the lower bound of the determinant:
\begin{equation*}\label{eq:L-Problem}\tag{L-Problem}
\overline{L}=\min_{P_N=(x,y)\in N}F(x,y):=\determinant{P^{}_N}{P_{B'}^1}{P_{B'}^2}=\left|
\begin{array}{cc}
P^{1}_{B'_x}-x & P^{2}_{B'_x}-x \\
P^{1}_{B'_y}-y & P^{2}_{B'_y}-y
\end{array}
\right|.
\end{equation*}

\subsubsection{Lower and upper bounds when the neighborhoods are line segments}
In this case, the segment whose endpoints are $P^1_{N_S}$ and $P^2_{N_S}$ can be expressed as the following convex set:
$$N=\{(x,y)\in\mathbb R^2:(x,y)=\mu P^1_{N}+(1-\mu)P^2_{N}, 0\leq\mu\leq1\}.$$
Since we are optimizing a linear function in a compact set we can conclude that the objective function in \eqref{eq:L-Problem} achieves its minimum and its maximum in the extreme points of $N_S$, that is, in $P^1_{N}$ and $P^2_{N}$. 

\subsubsection{Lower and upper bounds when the neighborhoods are ellipsoids}
The first case that is considered is the one when $N$ is an ellipse, that is, $N$ is represented by the following inequality:
$$N=\{(x,y)\in\mathbb R^2:ax^2+by^2+cxy+dx+ey+f\leq 0\}=$$
where $a, b, c, d, e, f$ are coefficients of the ellipse.
In an extended form, we need to find:
\begin{mini*}
{}{F(x, y)=\left|
\begin{array}{cc}
P^{1}_{B'_x}-x & P^{2}_{B'_x}-x \\
P^{1}_{B'_y}-y & P^{2}_{B'_y}-y
\end{array}
\right|=xP^{1}_{B'_y}-xP^{2}_{B'_y}+yP^{2}_{B'_x}-yP^{1}_{B'_x}+P^{1}_{B'_x}P^{2}_{B'_y}-P^{1}_{B'_y}P^{2}_{B'_x},}
{\label{eq:Example1}}{\tag{L-Ellipse}}
\addConstraint{ax^2+by^2+cxy+dx+ey+f\leq 0.}
\end{mini*}
Since we are minimizing a linear function in a convex set, we can conclude that the extreme points are located in the frontier, so we can use the Lagrangian function to compute these points.
$$F(x,y;\lambda)=xP^{1}_{B'_y}-xP^{2}_{B'_y}+yP^{2}_{B'_x}-yP^{1}_{B'_x}+P^{1}_{B'_x}P^{2}_{B'_y}-P^{1}_{B'_y}P^{2}_{B'_x}+\lambda(ax^2+by^2+cxy+dx+ey+f).$$

$$\nabla F(x,y;\lambda)=0\Longleftrightarrow
\left\{\begin{array}{rcll}
\frac{\partial F}{\partial x} & = & P^{1}_{B'_y}-P^{2}_{B'_y}+2ax\lambda+cy\lambda+d\lambda& =0,\\
\frac{\partial F}{\partial y} & = & P^{2}_{B'_x}-P^{1}_{B'_x}+2by\lambda+cx\lambda+e\lambda& =0,\\
\frac{\partial F}{\partial \lambda} & = & ax^2+by^2+cxy+dx+ey+f& =0.\\
\end{array}\right.$$
From the first two equations we can obtain:

$$\lambda = \frac{P^{2}_{B'_y}-P^{1}_{B'_y}}{2ax+cy+d}=\frac{P^{1}_{B'_x}-P^{2}_{B'_x}}{2by+cx+e}.$$
From this equality, we can express $y$ as a function of $x$:
\begin{align*}
(P^{2}_{B'_y}-P^{1}_{B'_y})(2by+cx+e)&=(P^{1}_{B'_x}-P^{2}_{B'_x})(2ax+cy+d),\\
y\left[2b(P^{2}_{B'_y}-P^{1}_{B'_y})-c(P^{1}_{B'_x}-P^{2}_{B'_x})\right]&=\left[2a(P^{1}_{B'_x}-P^{2}_{B'_x})-c(P^{2}_{B'_y}-P^{1}_{B'_y})\right]x+\left[d(P^{1}_{B'_x}-P^{2}_{B'_x})-e(P^{2}_{B'_y}-P^{1}_{B'_y})\right],\\
y&=\left[\frac{2a(P^{1}_{B'_x}-P^{2}_{B'_x})-c(P^{2}_{B'_y}-P^{1}_{B'_y})}{2b(P^{2}_{B'_y}-P^{1}_{B'_y})-c(P^{1}_{B'_x}-P^{2}_{B'_x})}\right]x+\left[\frac{d(P^{1}_{B'_x}-P^{2}_{B'_x})-e(P^{2}_{B'_y}-P^{1}_{B'_y})}{2b(P^{2}_{B'_y}-P^{1}_{B'_y})-c(P^{1}_{B'_x}-P^{2}_{B'_x})}\right],\\
y&=mx+n.
\end{align*}
Finally, we can substitute $y$ in the third equation to compute the value of $x$:
\begin{align*}
ax^2+by^2+cxy+dx+ey+f &= ax^2+b(mx+n)^2+cx(mx+n)+dx+e(mx+n)+f =\\
&= (a+bm^2+cm)x^2+(2bmn+cn+d+em)x+(n^2b+en+f)=0.
\end{align*}
By using the standard form of the solution of a quadratic equation:
\begin{align*}
x^{\pm}&=\frac{-(2bmn+cn+d+em)\pm\sqrt{(2bmn+cn+d+em)^2-4(a+bm^2+cm)(n^2b+en+f)}}{2(a+bm^2+cm)},\\
y^{\pm}&=mx^{\pm}+n.
\end{align*}
Hence, to compute the lower and upper bounds, we only need to evaluate $x^{\pm}$ and $y^{\pm}$ in the objective function and the lowest and highest value correspond to $\LS{P_N^{}}{P_{B'}^1}{P_{B'}^2}$ and $\US{P_N^{}}{P_{B'}^1}{P_{B'}^2}$, respectively.

\CV{
\section{Computational experiments}
The following section is devoted to study the behaviour of the formulations proposed in the Section \ref{section:formulations}. In the first subsection, the procedure of generating random instances is described. The second one details the configuration of the experiments that have been executed. The third subsection reports the results obtained in these experiments.

\subsection{Data generation}
%To generate the instances of our experiments, it is necessary that these instances verify the assumptions \ref{A1}-\ref{A4} stated in the Section \ref{section:description}. 

To generate the instances of our experiments, we will take into account the assumptions \ref{A1}-\ref{A4} stated in the Section \ref{section:description}. The idea is to generate line segments located in a general position without crossings and neighborhoods that do not intersect with these line segments. The following procedure describes how to construct the instances when the neighborhoods are balls.

\begin{algorithm}[H]
\caption{Generation of instances when the neighborhoods are balls}
\label{alg:Algorithm3}
\SetKwInOut{Input}{Initialization}\SetKwInOut{Output}{output}

 \Input{Let $|\mathcal N|$ be the number of neighborhoods to generate. Let $r_{\text{init}}=10$ be the half of the initial length of the barriers.
 Set $\mathcal N = \{\}$; $points=\{\}$; $\mathcal B=\{\overline{(0, 0)(100, 0)}, \overline{(100, 0)(100, 100)}, \overline{(100, 100)(0, 100)}, \overline{(0, 100)(0, 0)}\}$.}
% Set $LB=0$, $UB=+\infty$, $\bar z=z^0$.
Generate $|\mathcal N|$ points uniformly distributed in the square $[0, 100]^2$ and include them in $points$. \\
%\While{There does not exist a barrier that separates each pair of points}{
\For{$P, P'\in points$}{
	\If{$\overline{PP'}\cap B=\emptyset,\;\forall B\in\mathcal B$}{
		Compute $\overrightarrow{d}=\overrightarrow{PP'}$.\\
		Compute $M = P + \frac 1 2 \overrightarrow{d}$.\\
		Compute the unitary vector $\overrightarrow{n_u}$ perpendicular to $\overrightarrow{d}$.\\
		Set $r = r_{\text{init}}$. \\
		Generate the barrier $B(r)=\overline{P_B^{+} P_B^{-}}$ where
		$P_B^{\pm}=M \pm r\overrightarrow{n_u}.$\\
		\While{$B(r)\cap B'\neq\emptyset$ for some $B'\in\mathcal B$}{
			Set $r := r / 2$. \\
			Generate the barrier $B(r)$.
			}
		
		Include $B(r)$ in $\mathcal B$.
		
		
	}
}
\For{$P\in points$}{
Set $r_{\text{min}}=\min_{\{P_B\in B: B \in\mathcal B\}} d(P, P_B).$\\
Generate a random $radii$ uniformly distributed in the interval $\left[\frac 1 2 r_{\text{min}}, r_{\text{min}}\right]$.\\
Set the ball $N$ whose center is $P$ and radii is $radii$. \\
Include $N$ in $\mathcal N$.
}
%}
\end{algorithm}

To generate the instances when the neighborhoods are segments, we can take the balls built in the previous procedure and draw a random angle that determines which point and its diametrically opposite point are selected as the endpoints of the line segment.


\subsection{Configuration of the experiments}
Since no benchmark instances are available in the literature for this problem, we have generated ten instances with a number $|\mathcal N|\in\{5, 10, 20, 30, 50, 80\}$ of two typologies: balls and line segments. We have considered the cases with and without preprocessing the variables of the formulations as explained in Subsection \ref{subsection:preprocessing} and we have reported the average results. 

The formulations were coded in Python 3.9.2 and solved in Gurobi 9.1.2 \cite{GurobiOptimization2021} in a \textcolor{red}{AMD® Epyc 7402p 24-core processor}. A time limit of 1 hour was set in the experiments.

}
\subsection{Results of the experiments}

% \DeclareUnicodeCharacter{2212}{−}
% \usepgfplotslibrary{groupplots}
\usetikzlibrary{patterns,shapes.arrows}
\pgfplotsset{compat=newest}

\definecolor{darkslategray38}{RGB}{38,38,38}
\definecolor{darkslategray76}{RGB}{76,76,76}
\definecolor{lavender234234242}{RGB}{234,234,242}
\definecolor{peru20313699}{RGB}{203,136,99}
\definecolor{steelblue88116163}{RGB}{88,116,163}

\begin{figure}[h!]
    \centering
\begin{tikzpicture}[scale=0.9]

\definecolor{color0}{rgb}{0.917647058823529,0.917647058823529,0.949019607843137}
\definecolor{color1}{rgb}{0.347058823529412,0.458823529411765,0.641176470588235}
\definecolor{color2}{rgb}{0.798529411764706,0.536764705882353,0.389705882352941}

\begin{axis}[
axis background/.style={fill=color0},
axis line style={white},
tick align=outside,
title={Neighborhood = Segment},
x grid style={white},
xtick pos = left,
ytick pos = left,
xlabel={$|\mathcal N|$},
% xmajorticks=false,
xmin=-0.5, xmax=6.5,
xtick style={color=white!15!black},
xtick={0,1,2,3,4,5,6},
xticklabels={5, 10, 20, 30, 50, 80, 100},
y grid style={white},
ylabel={Runtime},
ymajorgrids,
% ymajorticks=false,
ymin=-0.05, ymax= 7200,
log basis y={10},
ymode = log,
ytick style={color=white!15!black},
%ytick={-0.2,0,0.2,0.4,0.6,0.8,1,1.2},
%yticklabels={−0.2,0.0,0.2,0.4,0.6,0.8,1.0,1.2},
legend pos = south east
]
\path [draw=darkslategray76, fill=steelblue88116163, semithick]
(axis cs:-0.396,1.32029229402542)
--(axis cs:-0.004,1.32029229402542)
--(axis cs:-0.004,2321.5214869976)
--(axis cs:-0.396,2321.5214869976)
--(axis cs:-0.396,1.32029229402542)
--cycle;
\path [draw=darkslategray76, fill=peru20313699, semithick]
(axis cs:0.004,0.867011845111847)
--(axis cs:0.396,0.867011845111847)
--(axis cs:0.396,1.71693199872971)
--(axis cs:0.004,1.71693199872971)
--(axis cs:0.004,0.867011845111847)
--cycle;
\path [draw=darkslategray76, fill=steelblue88116163, semithick]
(axis cs:0.604,22.0478292107582)
--(axis cs:0.996,22.0478292107582)
--(axis cs:0.996,3600.1058292985)
--(axis cs:0.604,3600.1058292985)
--(axis cs:0.604,22.0478292107582)
--cycle;
\path [draw=darkslategray76, fill=peru20313699, semithick]
(axis cs:1.004,5.35795325040817)
--(axis cs:1.396,5.35795325040817)
--(axis cs:1.396,11.317633330822)
--(axis cs:1.004,11.317633330822)
--(axis cs:1.004,5.35795325040817)
--cycle;
\path [draw=darkslategray76, fill=steelblue88116163, semithick]
(axis cs:1.604,336.735125482082)
--(axis cs:1.996,336.735125482082)
--(axis cs:1.996,2661.89892309904)
--(axis cs:1.604,2661.89892309904)
--(axis cs:1.604,336.735125482082)
--cycle;
\path [draw=darkslategray76, fill=peru20313699, semithick]
(axis cs:2.004,124.047738432884)
--(axis cs:2.396,124.047738432884)
--(axis cs:2.396,243.347989618778)
--(axis cs:2.004,243.347989618778)
--(axis cs:2.004,124.047738432884)
--cycle;
\path [draw=darkslategray76, fill=steelblue88116163, semithick]
(axis cs:2.604,758.177148163319)
--(axis cs:2.996,758.177148163319)
--(axis cs:2.996,3601.57422971725)
--(axis cs:2.604,3601.57422971725)
--(axis cs:2.604,758.177148163319)
--cycle;
\path [draw=darkslategray76, fill=peru20313699, semithick]
(axis cs:3.004,415.898770868778)
--(axis cs:3.396,415.898770868778)
--(axis cs:3.396,1199.34626537561)
--(axis cs:3.004,1199.34626537561)
--(axis cs:3.004,415.898770868778)
--cycle;
\path [draw=darkslategray76, fill=steelblue88116163, semithick]
(axis cs:3.604,3601.05894333124)
--(axis cs:3.996,3601.05894333124)
--(axis cs:3.996,3606.46183502674)
--(axis cs:3.604,3606.46183502674)
--(axis cs:3.604,3601.05894333124)
--cycle;
\path [draw=darkslategray76, fill=peru20313699, semithick]
(axis cs:4.004,2274.12813621759)
--(axis cs:4.396,2274.12813621759)
--(axis cs:4.396,3604.64131337404)
--(axis cs:4.004,3604.64131337404)
--(axis cs:4.004,2274.12813621759)
--cycle;
\path [draw=darkslategray76, fill=steelblue88116163, semithick]
(axis cs:4.604,3603.31649148464)
--(axis cs:4.996,3603.31649148464)
--(axis cs:4.996,3606.0530809164)
--(axis cs:4.604,3606.0530809164)
--(axis cs:4.604,3603.31649148464)
--cycle;
\path [draw=darkslategray76, fill=peru20313699, semithick]
(axis cs:5.004,3602.73281055689)
--(axis cs:5.396,3602.73281055689)
--(axis cs:5.396,3617.00508999825)
--(axis cs:5.004,3617.00508999825)
--(axis cs:5.004,3602.73281055689)
--cycle;
\path [draw=darkslategray76, fill=steelblue88116163, semithick]
(axis cs:5.604,3604.5290222168)
--(axis cs:5.996,3604.5290222168)
--(axis cs:5.996,3639.30546808243)
--(axis cs:5.604,3639.30546808243)
--(axis cs:5.604,3604.5290222168)
--cycle;
\path [draw=darkslategray76, fill=peru20313699, semithick]
(axis cs:6.004,3603.62441074848)
--(axis cs:6.396,3603.62441074848)
--(axis cs:6.396,3604.991119802)
--(axis cs:6.004,3604.991119802)
--(axis cs:6.004,3603.62441074848)
--cycle;
\path [draw=darkslategray76, fill=steelblue88116163, semithick]
(axis cs:6.604,3613.56903254986)
--(axis cs:6.996,3613.56903254986)
--(axis cs:6.996,3668.42798751593)
--(axis cs:6.604,3668.42798751593)
--(axis cs:6.604,3613.56903254986)
--cycle;
\path [draw=darkslategray76, fill=peru20313699, semithick]
(axis cs:7.004,3604.49900507927)
--(axis cs:7.396,3604.49900507927)
--(axis cs:7.396,3605.37177205086)
--(axis cs:7.004,3605.37177205086)
--(axis cs:7.004,3604.49900507927)
--cycle;
\draw[draw=darkslategray76,fill=steelblue88116163,line width=0.3pt] (axis cs:0,0) rectangle (axis cs:0,0);
\addlegendimage{ybar,ybar legend,draw=darkslategray76,fill=steelblue88116163,line width=0.3pt}
\addlegendentry{Without preprocessing}

\draw[draw=darkslategray76,fill=peru20313699,line width=0.3pt] (axis cs:0,0) rectangle (axis cs:0,0);
\addlegendimage{ybar,ybar legend,draw=darkslategray76,fill=peru20313699,line width=0.3pt}
\addlegendentry{With preprocessing}

\addplot [semithick, darkslategray76]
table {%
	-0.2 1.32029229402542
	-0.2 0.679740905761719
};
\addplot [semithick, darkslategray76]
table {%
	-0.2 2321.5214869976
	-0.2 3600.04041099548
};
\addplot [semithick, darkslategray76]
table {%
	-0.298 0.679740905761719
	-0.102 0.679740905761719
};
\addplot [semithick, darkslategray76]
table {%
	-0.298 3600.04041099548
	-0.102 3600.04041099548
};
\addplot [semithick, darkslategray76]
table {%
	0.2 0.867011845111847
	0.2 0.598104000091553
};
\addplot [semithick, darkslategray76]
table {%
	0.2 1.71693199872971
	0.2 2.45890307426453
};
\addplot [semithick, darkslategray76]
table {%
	0.102 0.598104000091553
	0.298 0.598104000091553
};
\addplot [semithick, darkslategray76]
table {%
	0.102 2.45890307426453
	0.298 2.45890307426453
};
\addplot [semithick, darkslategray76]
table {%
	0.8 22.0478292107582
	0.8 16.7804238796234
};
\addplot [semithick, darkslategray76]
table {%
	0.8 3600.1058292985
	0.8 3600.14186501503
};
\addplot [semithick, darkslategray76]
table {%
	0.702 16.7804238796234
	0.898 16.7804238796234
};
\addplot [semithick, darkslategray76]
table {%
	0.702 3600.14186501503
	0.898 3600.14186501503
};
\addplot [semithick, darkslategray76]
table {%
	1.2 5.35795325040817
	1.2 2.66339802742004
};
\addplot [semithick, darkslategray76]
table {%
	1.2 11.317633330822
	1.2 12.2503089904785
};
\addplot [semithick, darkslategray76]
table {%
	1.102 2.66339802742004
	1.298 2.66339802742004
};
\addplot [semithick, darkslategray76]
table {%
	1.102 12.2503089904785
	1.298 12.2503089904785
};
\addplot [semithick, darkslategray76]
table {%
	1.8 336.735125482082
	1.8 166.894387960434
};
\addplot [semithick, darkslategray76]
table {%
	1.8 2661.89892309904
	1.8 3600.55163097382
};
\addplot [semithick, darkslategray76]
table {%
	1.702 166.894387960434
	1.898 166.894387960434
};
\addplot [semithick, darkslategray76]
table {%
	1.702 3600.55163097382
	1.898 3600.55163097382
};
\addplot [semithick, darkslategray76]
table {%
	2.2 124.047738432884
	2.2 30.748733997345
};
\addplot [semithick, darkslategray76]
table {%
	2.2 243.347989618778
	2.2 380.830942869186
};
\addplot [semithick, darkslategray76]
table {%
	2.102 30.748733997345
	2.298 30.748733997345
};
\addplot [semithick, darkslategray76]
table {%
	2.102 380.830942869186
	2.298 380.830942869186
};
\addplot [black, mark=diamond*, mark size=2.5, mark options={solid,fill=darkslategray76}, only marks]
table {%
	2.2 426.554292917252
};
\addplot [semithick, darkslategray76]
table {%
	2.8 758.177148163319
	2.8 315.556146144867
};
\addplot [semithick, darkslategray76]
table {%
	2.8 3601.57422971725
	2.8 3601.94508290291
};
\addplot [semithick, darkslategray76]
table {%
	2.702 315.556146144867
	2.898 315.556146144867
};
\addplot [semithick, darkslategray76]
table {%
	2.702 3601.94508290291
	2.898 3601.94508290291
};
\addplot [semithick, darkslategray76]
table {%
	3.2 415.898770868778
	3.2 353.725795984268
};
\addplot [semithick, darkslategray76]
table {%
	3.2 1199.34626537561
	3.2 1215.04742217064
};
\addplot [semithick, darkslategray76]
table {%
	3.102 353.725795984268
	3.298 353.725795984268
};
\addplot [semithick, darkslategray76]
table {%
	3.102 1215.04742217064
	3.298 1215.04742217064
};
\addplot [black, mark=diamond*, mark size=2.5, mark options={solid,fill=darkslategray76}, only marks]
table {%
	3.2 2735.75876998901
	3.2 3209.93583416939
};
\addplot [semithick, darkslategray76]
table {%
	3.8 3601.05894333124
	3.8 3600.85991501808
};
\addplot [semithick, darkslategray76]
table {%
	3.8 3606.46183502674
	3.8 3607.48930191994
};
\addplot [semithick, darkslategray76]
table {%
	3.702 3600.85991501808
	3.898 3600.85991501808
};
\addplot [semithick, darkslategray76]
table {%
	3.702 3607.48930191994
	3.898 3607.48930191994
};
\addplot [semithick, darkslategray76]
table {%
	4.2 2274.12813621759
	4.2 2053.62032914162
};
\addplot [semithick, darkslategray76]
table {%
	4.2 3604.64131337404
	4.2 3607.03867697716
};
\addplot [semithick, darkslategray76]
table {%
	4.102 2053.62032914162
	4.298 2053.62032914162
};
\addplot [semithick, darkslategray76]
table {%
	4.102 3607.03867697716
	4.298 3607.03867697716
};
\addplot [semithick, darkslategray76]
table {%
	4.8 3603.31649148464
	4.8 3602.91880893707
};
\addplot [semithick, darkslategray76]
table {%
	4.8 3606.0530809164
	4.8 3606.30187892914
};
\addplot [semithick, darkslategray76]
table {%
	4.702 3602.91880893707
	4.898 3602.91880893707
};
\addplot [semithick, darkslategray76]
table {%
	4.702 3606.30187892914
	4.898 3606.30187892914
};
\addplot [black, mark=diamond*, mark size=2.5, mark options={solid,fill=darkslategray76}, only marks]
table {%
	4.8 3625.4150249958
	4.8 3645.45013308525
};
\addplot [semithick, darkslategray76]
table {%
	5.2 3602.73281055689
	5.2 3602.33397316933
};
\addplot [semithick, darkslategray76]
table {%
	5.2 3617.00508999825
	5.2 3624.0230820179
};
\addplot [semithick, darkslategray76]
table {%
	5.102 3602.33397316933
	5.298 3602.33397316933
};
\addplot [semithick, darkslategray76]
table {%
	5.102 3624.0230820179
	5.298 3624.0230820179
};
\addplot [semithick, darkslategray76]
table {%
	5.8 3604.5290222168
	5.8 3604.30805683136
};
\addplot [semithick, darkslategray76]
table {%
	5.8 3639.30546808243
	5.8 3645.38289093971
};
\addplot [semithick, darkslategray76]
table {%
	5.702 3604.30805683136
	5.898 3604.30805683136
};
\addplot [semithick, darkslategray76]
table {%
	5.702 3645.38289093971
	5.898 3645.38289093971
};
\addplot [semithick, darkslategray76]
table {%
	6.2 3603.62441074848
	6.2 3603.08717489243
};
\addplot [semithick, darkslategray76]
table {%
	6.2 3604.991119802
	6.2 3605.26269197464
};
\addplot [semithick, darkslategray76]
table {%
	6.102 3603.08717489243
	6.298 3603.08717489243
};
\addplot [semithick, darkslategray76]
table {%
	6.102 3605.26269197464
	6.298 3605.26269197464
};
\addplot [semithick, darkslategray76]
table {%
	6.8 3613.56903254986
	6.8 3605.69076514244
};
\addplot [semithick, darkslategray76]
table {%
	6.8 3668.42798751593
	6.8 3675.8382871151
};
\addplot [semithick, darkslategray76]
table {%
	6.702 3605.69076514244
	6.898 3605.69076514244
};
\addplot [semithick, darkslategray76]
table {%
	6.702 3675.8382871151
	6.898 3675.8382871151
};
\addplot [semithick, darkslategray76]
table {%
	7.2 3604.49900507927
	7.2 3603.45564484596
};
\addplot [semithick, darkslategray76]
table {%
	7.2 3605.37177205086
	7.2 3605.82377719879
};
\addplot [semithick, darkslategray76]
table {%
	7.102 3603.45564484596
	7.298 3603.45564484596
};
\addplot [semithick, darkslategray76]
table {%
	7.102 3605.82377719879
	7.298 3605.82377719879
};
\addplot [black, mark=diamond*, mark size=2.5, mark options={solid,fill=darkslategray76}, only marks]
table {%
	7.2 3658.00065898895
};
\addplot [semithick, darkslategray76]
table {%
	-0.396 10.1803375482559
	-0.004 10.1803375482559
};
\addplot [semithick, darkslategray76]
table {%
	0.004 1.51285195350647
	0.396 1.51285195350647
};
\addplot [semithick, darkslategray76]
table {%
	0.604 59.596302986145
	0.996 59.596302986145
};
\addplot [semithick, darkslategray76]
table {%
	1.004 9.26608741283417
	1.396 9.26608741283417
};
\addplot [semithick, darkslategray76]
table {%
	1.604 761.183493494987
	1.996 761.183493494987
};
\addplot [semithick, darkslategray76]
table {%
	2.004 188.488757133484
	2.396 188.488757133484
};
\addplot [semithick, darkslategray76]
table {%
	2.604 2137.81448185444
	2.996 2137.81448185444
};
\addplot [semithick, darkslategray76]
table {%
	3.004 755.287425994873
	3.396 755.287425994873
};
\addplot [semithick, darkslategray76]
table {%
	3.604 3606.29241991043
	3.996 3606.29241991043
};
\addplot [semithick, darkslategray76]
table {%
	4.004 3148.22736597061
	4.396 3148.22736597061
};
\addplot [semithick, darkslategray76]
table {%
	4.604 3603.71865308285
	4.996 3603.71865308285
};
\addplot [semithick, darkslategray76]
table {%
	5.004 3604.7077203989
	5.396 3604.7077203989
};
\addplot [semithick, darkslategray76]
table {%
	5.604 3606.53671908379
	5.996 3606.53671908379
};
\addplot [semithick, darkslategray76]
table {%
	6.004 3604.49580192566
	6.396 3604.49580192566
};
\addplot [semithick, darkslategray76]
table {%
	6.604 3661.83730447292
	6.996 3661.83730447292
};
\addplot [semithick, darkslategray76]
table {%
	7.004 3604.87411022186
	7.396 3604.87411022186
};
\end{axis}

\end{tikzpicture}
\begin{tikzpicture}[scale=0.9]

\definecolor{color0}{rgb}{0.917647058823529,0.917647058823529,0.949019607843137}
\definecolor{color1}{rgb}{0.347058823529412,0.458823529411765,0.641176470588235}
\definecolor{color2}{rgb}{0.798529411764706,0.536764705882353,0.389705882352941}

\begin{axis}[
axis background/.style={fill=color0},
axis line style={white},
tick align=outside,
title={Neighborhood = Ball},
x grid style={white},
xtick pos = left,
ytick pos = left,
xlabel={$|\mathcal N|$},
% xmajorticks=false,
xmin=-0.5, xmax=6.5,
xtick style={color=white!15!black},
xtick={0,1,2,3,4,5,6},
xticklabels={5, 10, 20, 30, 50, 80, 100},
y grid style={white},
ylabel={Runtime},
ymajorgrids,
% ymajorticks=false,
ymin=-0.05, ymax= 7200,
log basis y={10},
ymode = log,
ytick style={color=white!15!black},
%ytick={-0.2,0,0.2,0.4,0.6,0.8,1,1.2},
%yticklabels={−0.2,0.0,0.2,0.4,0.6,0.8,1.0,1.2},
legend pos = south east
]
\path [draw=darkslategray76, fill=steelblue88116163, semithick]
(axis cs:-0.396,8.35419070720673)
--(axis cs:-0.004,8.35419070720673)
--(axis cs:-0.004,24.3694642186165)
--(axis cs:-0.396,24.3694642186165)
--(axis cs:-0.396,8.35419070720673)
--cycle;
\path [draw=darkslategray76, fill=peru20313699, semithick]
(axis cs:0.004,1.88320851325989)
--(axis cs:0.396,1.88320851325989)
--(axis cs:0.396,2.84040707349777)
--(axis cs:0.004,2.84040707349777)
--(axis cs:0.004,1.88320851325989)
--cycle;
\path [draw=darkslategray76, fill=steelblue88116163, semithick]
(axis cs:0.604,240.028206408024)
--(axis cs:0.996,240.028206408024)
--(axis cs:0.996,626.305102109909)
--(axis cs:0.604,626.305102109909)
--(axis cs:0.604,240.028206408024)
--cycle;
\path [draw=darkslategray76, fill=peru20313699, semithick]
(axis cs:1.004,19.6992551088333)
--(axis cs:1.396,19.6992551088333)
--(axis cs:1.396,47.0685350894928)
--(axis cs:1.004,47.0685350894928)
--(axis cs:1.004,19.6992551088333)
--cycle;
\path [draw=darkslategray76, fill=steelblue88116163, semithick]
(axis cs:1.604,3600.64035725594)
--(axis cs:1.996,3600.64035725594)
--(axis cs:1.996,3600.77081692219)
--(axis cs:1.604,3600.77081692219)
--(axis cs:1.604,3600.64035725594)
--cycle;
\path [draw=darkslategray76, fill=peru20313699, semithick]
(axis cs:2.004,281.493759214878)
--(axis cs:2.396,281.493759214878)
--(axis cs:2.396,2610.16294920444)
--(axis cs:2.004,2610.16294920444)
--(axis cs:2.004,281.493759214878)
--cycle;
\path [draw=darkslategray76, fill=steelblue88116163, semithick]
(axis cs:2.604,3601.56696248055)
--(axis cs:2.996,3601.56696248055)
--(axis cs:2.996,3601.82251447439)
--(axis cs:2.604,3601.82251447439)
--(axis cs:2.604,3601.56696248055)
--cycle;
\path [draw=darkslategray76, fill=peru20313699, semithick]
(axis cs:3.004,1318.59078878164)
--(axis cs:3.396,1318.59078878164)
--(axis cs:3.396,3600.86907690764)
--(axis cs:3.004,3600.86907690764)
--(axis cs:3.004,1318.59078878164)
--cycle;
\path [draw=darkslategray76, fill=steelblue88116163, semithick]
(axis cs:3.604,3606.05540072918)
--(axis cs:3.996,3606.05540072918)
--(axis cs:3.996,3607.22088718414)
--(axis cs:3.604,3607.22088718414)
--(axis cs:3.604,3606.05540072918)
--cycle;
\path [draw=darkslategray76, fill=peru20313699, semithick]
(axis cs:4.004,3601.0776720047)
--(axis cs:4.396,3601.0776720047)
--(axis cs:4.396,3606.17869776487)
--(axis cs:4.004,3606.17869776487)
--(axis cs:4.004,3601.0776720047)
--cycle;
\path [draw=darkslategray76, fill=steelblue88116163, semithick]
(axis cs:4.604,3603.55773746967)
--(axis cs:4.996,3603.55773746967)
--(axis cs:4.996,3624.90419363976)
--(axis cs:4.604,3624.90419363976)
--(axis cs:4.604,3603.55773746967)
--cycle;
\draw[draw=darkslategray76,fill=steelblue88116163,line width=0.3pt] (axis cs:0,0) rectangle (axis cs:0,0);
\addlegendimage{ybar,ybar legend,draw=darkslategray76,fill=steelblue88116163,line width=0.3pt}
\addlegendentry{Without preprocessing}

\draw[draw=darkslategray76,fill=peru20313699,line width=0.3pt] (axis cs:0,0) rectangle (axis cs:0,0);
\addlegendimage{ybar,ybar legend,draw=darkslategray76,fill=peru20313699,line width=0.3pt}
\addlegendentry{With preprocessing}

\addplot [semithick, darkslategray76]
table {%
	-0.2 8.35419070720673
	-0.2 2.95881795883179
};
\addplot [semithick, darkslategray76]
table {%
	-0.2 24.3694642186165
	-0.2 30.9444210529327
};
\addplot [semithick, darkslategray76]
table {%
	-0.298 2.95881795883179
	-0.102 2.95881795883179
};
\addplot [semithick, darkslategray76]
table {%
	-0.298 30.9444210529327
	-0.102 30.9444210529327
};
\addplot [black, mark=diamond*, mark size=2.5, mark options={solid,fill=darkslategray76}, only marks]
table {%
	-0.2 3600.03402090073
};
\addplot [semithick, darkslategray76]
table {%
	0.2 1.88320851325989
	0.2 1.39969992637634
};
\addplot [semithick, darkslategray76]
table {%
	0.2 2.84040707349777
	0.2 2.98832607269287
};
\addplot [semithick, darkslategray76]
table {%
	0.102 1.39969992637634
	0.298 1.39969992637634
};
\addplot [semithick, darkslategray76]
table {%
	0.102 2.98832607269287
	0.298 2.98832607269287
};
\addplot [black, mark=diamond*, mark size=2.5, mark options={solid,fill=darkslategray76}, only marks]
table {%
	0.2 12.9982089996338
	0.2 4.92047095298767
};
\addplot [semithick, darkslategray76]
table {%
	0.8 240.028206408024
	0.8 59.0895512104034
};
\addplot [semithick, darkslategray76]
table {%
	0.8 626.305102109909
	0.8 628.835431814194
};
\addplot [semithick, darkslategray76]
table {%
	0.702 59.0895512104034
	0.898 59.0895512104034
};
\addplot [semithick, darkslategray76]
table {%
	0.702 628.835431814194
	0.898 628.835431814194
};
\addplot [black, mark=diamond*, mark size=2.5, mark options={solid,fill=darkslategray76}, only marks]
table {%
	0.8 3600.14500808716
	0.8 3600.15608620644
};
\addplot [semithick, darkslategray76]
table {%
	1.2 19.6992551088333
	1.2 10.7248089313507
};
\addplot [semithick, darkslategray76]
table {%
	1.2 47.0685350894928
	1.2 87.2862269878387
};
\addplot [semithick, darkslategray76]
table {%
	1.102 10.7248089313507
	1.298 10.7248089313507
};
\addplot [semithick, darkslategray76]
table {%
	1.102 87.2862269878387
	1.298 87.2862269878387
};
\addplot [semithick, darkslategray76]
table {%
	1.8 3600.64035725594
	1.8 3600.53358101845
};
\addplot [semithick, darkslategray76]
table {%
	1.8 3600.77081692219
	1.8 3600.94052481651
};
\addplot [semithick, darkslategray76]
table {%
	1.702 3600.53358101845
	1.898 3600.53358101845
};
\addplot [semithick, darkslategray76]
table {%
	1.702 3600.94052481651
	1.898 3600.94052481651
};
\addplot [black, mark=diamond*, mark size=2.5, mark options={solid,fill=darkslategray76}, only marks]
table {%
	1.8 3600.37721896172
	1.8 3601.62018203735
};
\addplot [semithick, darkslategray76]
table {%
	2.2 281.493759214878
	2.2 140.541501998901
};
\addplot [semithick, darkslategray76]
table {%
	2.2 2610.16294920444
	2.2 3600.50284981728
};
\addplot [semithick, darkslategray76]
table {%
	2.102 140.541501998901
	2.298 140.541501998901
};
\addplot [semithick, darkslategray76]
table {%
	2.102 3600.50284981728
	2.298 3600.50284981728
};
\addplot [semithick, darkslategray76]
table {%
	2.8 3601.56696248055
	2.8 3601.54320001602
};
\addplot [semithick, darkslategray76]
table {%
	2.8 3601.82251447439
	2.8 3602.09848809242
};
\addplot [semithick, darkslategray76]
table {%
	2.702 3601.54320001602
	2.898 3601.54320001602
};
\addplot [semithick, darkslategray76]
table {%
	2.702 3602.09848809242
	2.898 3602.09848809242
};
\addplot [black, mark=diamond*, mark size=2.5, mark options={solid,fill=darkslategray76}, only marks]
table {%
	2.8 3246.53203010559
	2.8 3601.14429211617
};
\addplot [semithick, darkslategray76]
table {%
	3.2 1318.59078878164
	3.2 314.652143955231
};
\addplot [semithick, darkslategray76]
table {%
	3.2 3600.86907690764
	3.2 3601.7613158226
};
\addplot [semithick, darkslategray76]
table {%
	3.102 314.652143955231
	3.298 314.652143955231
};
\addplot [semithick, darkslategray76]
table {%
	3.102 3601.7613158226
	3.298 3601.7613158226
};
\addplot [semithick, darkslategray76]
table {%
	3.8 3606.05540072918
	3.8 3605.59163999557
};
\addplot [semithick, darkslategray76]
table {%
	3.8 3607.22088718414
	3.8 3607.99425077438
};
\addplot [semithick, darkslategray76]
table {%
	3.702 3605.59163999557
	3.898 3605.59163999557
};
\addplot [semithick, darkslategray76]
table {%
	3.702 3607.99425077438
	3.898 3607.99425077438
};
\addplot [black, mark=diamond*, mark size=2.5, mark options={solid,fill=darkslategray76}, only marks]
table {%
	3.8 3600.84237909317
};
\addplot [semithick, darkslategray76]
table {%
	4.2 3601.0776720047
	4.2 3600.90990781784
};
\addplot [semithick, darkslategray76]
table {%
	4.2 3606.17869776487
	4.2 3611.74287700653
};
\addplot [semithick, darkslategray76]
table {%
	4.102 3600.90990781784
	4.298 3600.90990781784
};
\addplot [semithick, darkslategray76]
table {%
	4.102 3611.74287700653
	4.298 3611.74287700653
};
\addplot [semithick, darkslategray76]
table {%
	4.8 3603.55773746967
	4.8 3602.58421516419
};
\addplot [semithick, darkslategray76]
table {%
	4.8 3624.90419363976
	4.8 3628.28486394882
};
\addplot [semithick, darkslategray76]
table {%
	4.702 3602.58421516419
	4.898 3602.58421516419
};
\addplot [semithick, darkslategray76]
table {%
	4.702 3628.28486394882
	4.898 3628.28486394882
};
\addplot [semithick, darkslategray76]
table {%
	-0.396 10.4243239164352
	-0.004 10.4243239164352
};
\addplot [semithick, darkslategray76]
table {%
	0.004 2.20789897441864
	0.396 2.20789897441864
};
\addplot [semithick, darkslategray76]
table {%
	0.604 432.304369449615
	0.996 432.304369449615
};
\addplot [semithick, darkslategray76]
table {%
	1.004 34.7487840652466
	1.396 34.7487840652466
};
\addplot [semithick, darkslategray76]
table {%
	1.604 3600.69731390476
	1.996 3600.69731390476
};
\addplot [semithick, darkslategray76]
table {%
	2.004 623.691154956818
	2.396 623.691154956818
};
\addplot [semithick, darkslategray76]
table {%
	2.604 3601.73447346687
	2.996 3601.73447346687
};
\addplot [semithick, darkslategray76]
table {%
	3.004 1822.5938924551
	3.396 1822.5938924551
};
\addplot [semithick, darkslategray76]
table {%
	3.604 3606.69710206985
	3.996 3606.69710206985
};
\addplot [semithick, darkslategray76]
table {%
	4.004 3603.10646545887
	4.396 3603.10646545887
};
\addplot [semithick, darkslategray76]
table {%
	4.604 3613.55721092224
	4.996 3613.55721092224
};
\end{axis}

\end{tikzpicture}


\caption{Runtime of the model \eqref{form:H-TSPN} without and with preprocessing when the neighborhoods are segments and balls.}
\label{fig:Fig4}
\end{figure}


\end{document}